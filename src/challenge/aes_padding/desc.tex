% Copyright (C) 2017 Daniel Page <dan@phoo.org>
%
% Use of this source code is restricted per the CC BY-SA license, a copy of
% which can be found via http://creativecommons.org (and should be included 
% as LICENSE.txt within the associated archive or repository).

\documentclass[crop={false},multi={true},tikz={true}]{standalone}

\ifstandalone
\usepackage{libbuild-style}
\usepackage{libbuild-macro}
\fi

\ifstandalone
\addbibresource{libbuild.bib}
\fi

\begin{document}

% -----------------------------------------------------------------------------

\ifstandalone
\author{\url{http://www.github.com/danpage/scale}}
\title{\Large SCALE challenge description : \lstinline[language={bash}]|\$\{ARCHIVE\}/CONF(ARCHIVE_PATH,CID)|}
\date{\today}

\maketitle
\fi

% -----------------------------------------------------------------------------

\leveldown{Background}

Imagine you encounter a server, denoted $\PARTY{D}$, 
which acts as a storage repository for sensor data.  An installation of $50$ 
or so IoT-class sensor nodes regularly transmit encrypted packets of data to
$\PARTY{D}$, which stores them for later analysis (iff. they are deemed to be
valid).  Having already captured some encrypted network traffic transmitted 
from a particular sensor node to the server, you are tasked with recovering 
the underlying data.
By leveraging access to the network, 
an attacker $\PARTY{E}$ can interact with $\PARTY{D}$ as follows:

\begin{center}
% Copyright (C) 2017 Daniel Page <dan@phoo.org>
%
% Use of this source code is restricted per the CC BY-SA license, a copy of
% which can be found via http://creativecommons.org (and should be included 
% as LICENSE.txt within the associated archive or repository).

\begin{tikzpicture}

% =============================================================================

% configure
 \tikzset{node distance={6cm},party/.style={rectangle,minimum width={2.0cm},minimum height={2.0cm}}}

% adversary and target device
 \node [                party] (E) {$\PARTY{E}$} ;
 \node [right of=E,draw,party] (T) {$\PARTY{T}$} ;

% embedded material
 \node at (T.south east) [anchor=south east,draw,rectangle] {\tiny $\PRI{\CHAL{k}}$} ;

%   direct input and output
 \draw [       >=stealth,->] (E.30)    to              
                             node [above]             {$\PUB{m}$} 
                             (E.30    -| T.west)  ;
 \draw [       >=stealth,<-] (E.330)   to              
                             node [below]             {$\PUB{c}$} 
                             (E.330   -| T.west)  ;

% indirect input and output
%\draw [dashed,>=stealth,<-] (T.north) to [bend right] 
%                            node [above]             {} 
%                            (T.north -| E.north) ;
 \draw [dashed,>=stealth,->] (T.south) to [bend left]  
                             node [below]             {$\LEAK  \models \mbox{execution latency}$} 
                             (T.south -| E.south) ;

% computation
 \draw [       >=stealth,->] (T.30)    to [bend left,in=90,out=90,looseness=2] 
                             node [right,anchor=west] {$\PUB{c} = \SCOPE{\ID{DES}}{\ALG{Enc}}( \PRI{\CHAL{k}}, \PUB{m} )$}
                             (T.330) ;

% =============================================================================

\end{tikzpicture}

\end{center}

\noindent
That is, in each interaction $\PARTY{E}$ can (adaptively) send 
a chosen AES ciphertext
to $\PARTY{D}$; the device

\begin{enumerate}
\item decrypts $\PUB{c}$, i.e., computes
      \[
      \PRI{m} \CONS \PRI{\TAG} \CONS \PRI{\PAD} = \SCOPE{\ID{AES-CBC}}{\ALG{Dec}}( \PRI{\CHAL{k}_1}, \PUB{iv}, \PUB{c} )
      \]
      then
\item checks whether $\PRI{\PAD}$ is valid, 
      aborting immediately  and producing result code $1$ if not,
      then
\item checks whether $\PRI{\TAG}$ is valid, i.e., whether 
      \[
      \SCOPE{\ID{HMAC-SHA-1}}{\ALG{Ver}}( \PRI{\CHAL{k}_2}, \PRI{m}, \PRI{\TAG} ) = \TRUE ,
      \]
      aborting immediately  and producing result code $2$ if not,
      then
\item processes $\PRI{m}$
                            and produces  result code $0$.
\end{enumerate}

\noindent
Note that the associated  plaintext is {\em not} produced explicitly as
output.

\levelstay{Materials}

\leveldown{\lstinline[language={bash}]|\$\{ARCHIVE\}/CONF(ARCHIVE_PATH,CID)/\$\{USER\}.D|}

This executable simulates the attack target $\PARTY{D}$.  When executed it 
reads the following input

\begin{itemize}
\item $\PUB{l}$,
      a  length
      (represented as a                       decimal integer string),
\item $\PUB{iv}$,
      a      ${1}$-block AES-CBC initialisation vector
      (represented as a  length-prefixed, hexadecimal octet   string),
      and
\item $\PUB{c}$,
      an $\PUB{l}$-block AES-CBC ciphertext
      (represented as a  length-prefixed, hexadecimal octet   string),
\end{itemize}

\noindent
from \lstinline[language={bash}]{stdin} and writes the following output

\begin{itemize}
\item $\LEAK$,
      a  result code
      (represented as a                       decimal integer string),
\end{itemize}

\noindent
to \lstinline[language={bash}]{stdout}, in both cases with one field per 
line.  Execution continues this way, i.e., by repeatedly reading input 
then writing output, until it is forcibly terminated (or crashes).  
Note that:

\begin{itemize}
\item To avoid timing attacks, $\PARTY{D}$ makes use of a high-performance
      yet constant-time AES-128 implementation (operated in CBC mode) that
      is derived from techniques in~\cite{SCALE:KasSch:09};
      this clearly implies $128$-bit block and cipher key lengths.  
\item Keep in mind some limits, namely $0 \leq \PUB{l} < 256$, on maximum
      plaintext and/or ciphertext length.
\item The padding scheme used matches that of TLS.  This means
      \[
      \PRI{\PAD} = \LIST{ \underbrace{ \alpha, \alpha, \ldots, \alpha }_{\mbox{$\alpha$ + 1 octets}} }
      \]
      is a sequence of $\alpha + 1$ octets, each of whose value is $\alpha$
      and where $0 \leq \alpha < 256$, constructed st. the AES-CBC input is
      always a multiple of the block size.
\end{itemize}

\levelstay{\lstinline[language={bash}]|\$\{ARCHIVE\}/CONF(ARCHIVE_PATH,CID)/\$\{USER\}.conf|}

This file represents a set of attack parameters, with everything (e.g.,
all public values) $\PARTY{E}$ has access to by default.  It contains 

\begin{itemize}
\item $\PUB{\CHAL{iv}}$,
      a  ${1}$-block AES-CBC initialisation vector
      (represented as a  length-prefixed, hexadecimal octet   string),
      and
\item $\PUB{\CHAL{c}}$,
      a  ${1}$-block AES-CBC ciphertext 
      (represented as a  length-prefixed, hexadecimal octet   string),
      corresponding to an encryption of some unknown plaintext 
      $\PRI{\CHAL{m}}$ (using $\PUB{\CHAL{iv}}$),
\end{itemize}

\noindent
with one field per line.
More specifically, this represents the previously captured network traffic
whose decryption, i.e., recovery of $\PRI{\CHAL{m}}$, is the task at hand.
\IfStrEqCase{CONF(CHALLENGE,CID)}{%
  {0}{Keep in mind that $\PRI{\CHAL{m}}$ will 
      include the SHA-1 hash          of \lstinline[language={bash}]{$\{USER\}}
      as the least-significant octets: 
      this allows candidate decryptions to be checked for validity.
  }%
  {1}{Keep in mind that $\PRI{\CHAL{m}}$ will 
      include an ASCII representation of \lstinline[language={bash}]{$\{USER\}} 
      as the least-significant octets: 
      this allows candidate decryptions to be checked for validity.
  }%
  {2}{Keep in mind that $\PRI{\CHAL{m}}$ will have been
      be generated entirely at random:
      this means checking validity of candidate decryptions is more 
      difficult than it would be otherwise.
  }%
}%

\levelup{Tasks}

\begin{enumerate}
\item Write a program that simulates the adversary $\PARTY{E}$ by attacking
      the simulated target, or, more specifically, that recovers the target 
      material $\PRI{\CHAL{m}}$.  
      When executed using a command of the form
      \[
      \mbox{\lstinline[language={bash}]|./attack \$\{USER\}.D \$\{USER\}.conf|}
      \]
      the attack should be invoked on the simulated target named (not some
      hard-coded alternative).  Use \lstinline[language={bash}]{stdout} to 
      print 
      a) any intermediate output you deem relevant, followed finally by 
      b) two lines which clearly detail the target material recovered plus
         the total number of interactions with the attack target.
\item Answer the exam-style questions in 
      \lstinline[language={bash}]|${ARCHIVE}/CONF(ARCHIVE_PATH,CID)/${USER}.exam|.
\end{enumerate}

% -----------------------------------------------------------------------------

\ifstandalone
\printbibliography
\fi

% -----------------------------------------------------------------------------

\ifstandalone
\appendix 
\section{Representation and conversion}

\subsection{Integers}

\paragraph{Concept.}
\input{rep-integer_concept.tex}
\paragraph{Example.}
Consider the following Python $3.x$ program
 
\begin{lstlisting}[language={Python}]
t_0 = '09080706050403020100'

t_1 = int( t_0, 10 )
t_2 = int( t_0, 16 )

t_3 = ( '{0:d}'.format( t_2 ) )
t_4 = ( '{0:X}'.format( t_2 ) )

t_5 = ( '{0:X}'.format( t_2 ) ).zfill( 20 )

print( 'type( t_0 ) = {0!s:14s} t_0 = {1!s}'.format( type( t_0 ), t_0 ) )
print( 'type( t_1 ) = {0!s:14s} t_1 = {1!s}'.format( type( t_1 ), t_1 ) )
print( 'type( t_2 ) = {0!s:14s} t_2 = {1!s}'.format( type( t_2 ), t_2 ) )
print( 'type( t_3 ) = {0!s:14s} t_3 = {1!s}'.format( type( t_3 ), t_3 ) )
print( 'type( t_4 ) = {0!s:14s} t_4 = {1!s}'.format( type( t_4 ), t_4 ) )
print( 'type( t_5 ) = {0!s:14s} t_5 = {1!s}'.format( type( t_5 ), t_5 ) )
\end{lstlisting}
 
\noindent
which, when executed, produces
 
\begin{lstlisting}[language={bash}]
type( t_0 ) = <class 'str'>  t_0 = 09080706050403020100
type( t_1 ) = <class 'int'>  t_1 = 9080706050403020100
type( t_2 ) = <class 'int'>  t_2 = 42649378395939397566720
type( t_3 ) = <class 'str'>  t_3 = 42649378395939397566720
type( t_4 ) = <class 'str'>  t_4 = 9080706050403020100
type( t_5 ) = <class 'str'>  t_5 = 09080706050403020100
\end{lstlisting}

\noindent
as output.  This is intended to illustrate that
 
\begin{itemize}
\item \lstinline[language={Python}]|t_0| is an
      integer string
      (i.e., a sequence of characters),
\item \lstinline[language={Python}]|t_1| 
      and 
      \lstinline[language={Python}]|t_2|
      are conversions of 
      \lstinline[language={Python}]|t_0| 
      into
      integers,
      using decimal and hexadecimal respectively,
      and
\item \lstinline[language={Python}]|t_3| 
      and
      \lstinline[language={Python}]|t_4|
      are conversions of 
      \lstinline[language={Python}]|t_2| 
      into
      strings
      (i.e., a sequence of characters),
      using decimal and hexadecimal respectively.
\end{itemize}
 
\noindent
Note that 
\lstinline[language={Python}]|t_0| 
and 
\lstinline[language={Python}]|t_4| 
do not match: 
the conversion ignored the left-most zero character, because it is not
significant wrt. the associated integer value.  
If/when this issue is problematic, it can be resolved by using the
\lstinline[language={Python}]|zfill| 
function to left-fill the string (with zero characters) so it is of the 
required length.
\lstinline[language={Python}]|t_0| 
and
\lstinline[language={Python}]|t_5| 
do     match,
for example,
because the latter has been filled to ensure an overall length of $20$ 
characters.


\subsection{Octet strings}

\paragraph{Concept.}
\input{rep-octet_concept.tex}
\paragraph{Example.}
Consider the following Python program

\begin{lstlisting}[language={Python}]
import binascii

def str2seq( x ) :
  return          [ ord( t ) for t in x ]

def seq2str( x ) :
  return ''.join( [ chr( t ) for t in x ] )

def octetstr2str( x ) :
  t = x.split( ':' ) ; n = int( t[ 0 ], 16 ) ; x = binascii.a2b_hex( t[ 1 ] )

  if( n != len( x ) ) :
    raise ValueError
  else :
    return x

def str2octetstr( x ) :
  return ( '%02X' % ( len( x ) ) ) + ':' + ( binascii.b2a_hex( x ) )

t_0 = '\xDE\xAD\xBE\xEF' ; t_1 = [ 0xDE, 0xAD, 0xBE, 0xEF ]

t_2 = str2octetstr(          t_0   )
t_3 = str2octetstr( seq2str( t_1 ) )

print "type( t_0 ) = %-18s t_0 = %s" % ( type( t_0 ), repr( t_0 ) )
print "type( t_1 ) = %-18s t_1 = %s" % ( type( t_1 ), repr( t_1 ) )
print "type( t_2 ) = %-18s t_2 = %s" % ( type( t_2 ), repr( t_2 ) )
print "type( t_3 ) = %-18s t_3 = %s" % ( type( t_3 ), repr( t_3 ) )

t_4 = '04:8BADF00D'

t_5 =          octetstr2str( t_4 )
t_6 = str2seq( octetstr2str( t_4 ) )

print "type( t_4 ) = %-18s t_4 = %s" % ( type( t_4 ), repr( t_4 ) )
print "type( t_5 ) = %-18s t_5 = %s" % ( type( t_5 ), repr( t_5 ) )
print "type( t_6 ) = %-18s t_6 = %s" % ( type( t_6 ), repr( t_6 ) )
\end{lstlisting}

\noindent
which, when executed, produces

\begin{lstlisting}[language={bash}]
type( t_0 ) = <type 'str'>       t_0 = '\xde\xad\xbe\xef'
type( t_1 ) = <type 'list'>      t_1 = [222, 173, 190, 239]
type( t_2 ) = <type 'str'>       t_2 = '04:deadbeef'
type( t_3 ) = <type 'str'>       t_3 = '04:deadbeef'
type( t_4 ) = <type 'str'>       t_4 = '04:8BADF00D'
type( t_5 ) = <type 'str'>       t_5 = '\x8b\xad\xf0\r'
type( t_6 ) = <type 'list'>      t_6 = [139, 173, 240, 13]
\end{lstlisting}

\noindent
as output.  This is intended to illustrate that

\begin{itemize}

\item If you start with 
      byte string (or array)
      \lstinline[language={Python}]|t_0|, 
      then 
                             \lstinline[language={Python}]|str2octetstr|
      convert this into the 
      octet string,
      you expect that
      $\mbox{\lstinline[language={Python}]|t_2|} = \mbox{\lstinline[language={Python}]|04:DEADBEEF|}$
      implies 
      $n = \RADIX{04}{16} = \RADIX{04}{10}$
      and
      $x = \LIST{ \RADIX{DE}{16}, \RADIX{AD}{16}, \RADIX{BE}{16}, \RADIX{EF}{16} }$
      st.
      $\mbox{\lstinline[language={Python}]|t_0[ 0 ]|} = \RADIX{DE}{16} = x_0$.

\item If you start with 
      byte sequence 
      \lstinline[language={Python}]|t_1|, 
      then 
      \lstinline[language={Python}]|seq2str| and \lstinline[language={Python}]|str2octetstr|
      convert this into the 
      octet string,
      you expect that
      $\mbox{\lstinline[language={Python}]|t_3|} = \mbox{\lstinline[language={Python}]|04:DEADBEEF|}$
      implies 
      $n = \RADIX{04}{16} = \RADIX{04}{10}$
      and
      $x = \LIST{ \RADIX{DE}{16}, \RADIX{AD}{16}, \RADIX{BE}{16}, \RADIX{EF}{16} }$
      st.
      $\mbox{\lstinline[language={Python}]|t_1[ 1 ]|} = \RADIX{AD}{16} = x_1$.

\item If you start with 
      octet string  
      \lstinline[language={Python}]|t_4|, 
      then 
      \lstinline[language={Python}]|octetstr2str|
      convert this into the byte array,    
      you expect that
      $\mbox{\lstinline[language={Python}]|t_4|} = \mbox{\lstinline[language={Python}]|04:8BADF00D|}$
      implies 
      $n = \RADIX{04}{16} = \RADIX{04}{10}$
      and
      $x = \LIST{ \RADIX{8B}{16}, \RADIX{AD}{16}, \RADIX{F0}{16}, \RADIX{0D}{16} }$
      st.
      $\mbox{\lstinline[language={Python}]|t_5[ 2 ]|} = \RADIX{F0}{16} = x_2$.

\item If you start with  
      octet string  
      \lstinline[language={Python}]|t_4|, 
      then 
      \lstinline[language={Python}]|octetstr2str| and \lstinline[language={Python}]|str2seq|
      convert this into the byte sequence, 
      you expect that
      $\mbox{\lstinline[language={Python}]|t_4|} = \mbox{\lstinline[language={Python}]|04:8BADF00D|}$
      implies 
      $n = \RADIX{04}{16} = \RADIX{04}{10}$
      and
      $x = \LIST{ \RADIX{8B}{16}, \RADIX{AD}{16}, \RADIX{F0}{16}, \RADIX{0D}{16} }$
      st.
      $\mbox{\lstinline[language={Python}]|t_6[ 3 ]|} = \RADIX{0D}{16} = x_3$.

\end{itemize}

\fi

% -----------------------------------------------------------------------------

\end{document}
