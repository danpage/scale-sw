% Copyright (C) 2017 Daniel Page <dan@phoo.org>
%
% Use of this source code is restricted per the CC BY-SA license, a copy of
% which can be found via http://creativecommons.org (and should be included 
% as LICENSE.txt within the associated archive or repository).

% =============================================================================

\leveldown{Background}

\DESCINTRO[leak+fault]
{a private key}
{a message}
{which signs the message using the under the private key}
{a signature}

% -----------------------------------------------------------------------------

\levelstay{Material}

\leveldown{\lstinline[language={bash}]|\$\{ARCHIVE\}/CONF(ARCHIVE_PATH,CID)/\$\{USER\}.D|}

\DESCMATERIALD{%
\item $\FAULT$
      a  fault specification
      (represented as a                       decimal integer string; the format and semantics of this field are explained further below),
\item $\PUB{m}$,
      a message
      (represented as a                   hexadecimal integer string),
}{
\item $\LEAK$,
      an execution time measured in clock cycles
      (represented as a                       decimal integer string),
      and
\item $\PUB{\SIG}$,
      an RSA signature on $m$
      (represented as a                   hexadecimal integer string),
}%
Note that:

\begin{itemize}
\item $\PARTY{D}$ uses a CRT-based implementation of RSA which is computed by
an external coprocessor which can perform basic arithmetic on large integers.

\item $\PARTY{E}$
      can induce a fault, via a ``glitch'' in the clock signal supplied to
      $\PARTY{D}$;
      the fault is controlled using $\FAULT$, in the sense that
      \[
      \begin{array}{c@{\;}c@{\;}c c c@{\;}l}
      \FAULT &<   & 0 &\Rightarrow& \mbox{no} & \mbox{fault induced                        } \\
      \FAULT &\geq& 0 &\Rightarrow&           & \mbox{fault induced in clock cycle $\FAULT$} \\
      \end{array}
      \]
      Through analysis of the fault model, it is known that inducing a fault 
      will impact the co-processor ALU; if a fault is induced in clock cycle
      $\FAULT$, computation by (or output of) the ALU in that clock cycle is
      corrupted (or randomised).
\end{itemize}

\levelstay{\lstinline[language={bash}]|\$\{ARCHIVE\}/CONF(ARCHIVE_PATH,CID)/\$\{USER\}.conf|}

\DESCMATERIALCONF{%
\item $\PUB{\CHAL{N}}$,     
      an RSA        modulus
      (represented as a                   hexadecimal integer string),
\item $\PUB{\CHAL{e}}$,
      an RSA        public exponent
      (represented as a                   hexadecimal integer string),
      st. $\PUB{\CHAL{e}} \cdot \PRI{\CHAL{d}} \equiv 1 \pmod{\PRI{\Phi{(\CHAL{N})}}}$,
}%
Note that
$
\TUPLE{ \PUB{\CHAL{N}}, \PUB{\CHAL{e}} }
$
is
a     (known) RSA public  key 
associated with 
$
\TUPLE{ \PUB{\CHAL{N}}, \PRI{\CHAL{d}} } ,
$
i.e.,
the (unknown) RSA private key embedded in $\PARTY{D}$.

% -----------------------------------------------------------------------------

\levelup{Tasks}

\begin{enumerate}
\item \DESCTASKIMPL
      {$\PRI{\CHAL{d}}$}
      {\mbox{\lstinline[language={bash}]|./attack \$\{USER\}.D \$\{USER\}.conf|}}
\item \DESCTASKEXAM
      {\mbox{\lstinline[language={bash}]|\$\{ARCHIVE\}/CONF(ARCHIVE_PATH,CID)/\$\{USER\}.exam|}}
\end{enumerate}

% =============================================================================
