% Copyright (C) 2017 Daniel Page <dan@phoo.org>
%
% Use of this source code is restricted per the CC BY-SA license, a copy of
% which can be found via http://creativecommons.org (and should be included 
% as LICENSE.txt within the associated archive or repository).

% =============================================================================

\leveldown{Background}

Imagine you are tasked with attacking a device, denoted 
$\PARTY{T}$, 
which houses an $8$-bit Intel $8051$ micro-processor.  More specifically,
$\PARTY{T}$ 
represents an ISO/IEC 7816 compliant contact-based smart-card used within 
larger devices as a cryptographic co-processor module: 
using a standardised protocol (or API), it supports secure key generation 
and storage plus off-load of some cryptographic operations.
\DESCINTRO[leak+fault]
{% Copyright (C) 2017 Daniel Page <dan@phoo.org>
%
% Use of this source code is restricted per the CC BY-SA license, a copy of
% which can be found via http://creativecommons.org (and should be included 
% as LICENSE.txt within the associated archive or repository).

\begin{tikzpicture}

% =============================================================================

% configure
 \tikzset{node distance={6cm},party/.style={rectangle,minimum width={2.0cm},minimum height={2.0cm}}}

% adversary and target device
 \node [                party] (E) {$\PARTY{E}$} ;
 \node [right of=E,draw,party] (T) {$\PARTY{T}$} ;

% embedded material
 \node at (T.south east) [anchor=south east,draw,rectangle] {\tiny $\PRI{\CHAL{k}}$} ;

%   direct input and output
 \draw [       >=stealth,->] (E.30)    to              
                             node [above]             {$\PUB{m}$} 
                             (E.30    -| T.west)  ;
 \draw [       >=stealth,<-] (E.330)   to              
                             node [below]             {$\PUB{c}$} 
                             (E.330   -| T.west)  ;

% indirect input and output
%\draw [dashed,>=stealth,<-] (T.north) to [bend right] 
%                            node [above]             {} 
%                            (T.north -| E.north) ;
 \draw [dashed,>=stealth,->] (T.south) to [bend left]  
                             node [below]             {$\LEAK  \models \mbox{execution latency}$} 
                             (T.south -| E.south) ;

% computation
 \draw [       >=stealth,->] (T.30)    to [bend left,in=90,out=90,looseness=2] 
                             node [right,anchor=west] {$\PUB{c} = \SCOPE{\ID{DES}}{\ALG{Enc}}( \PRI{\CHAL{k}}, \PUB{m} )$}
                             (T.330) ;

% =============================================================================

\end{tikzpicture}
}
{a private key}
{a message}
{i.e. it signs the message using the private key}
{a signature}
Note that
$\PARTY{E}$ 
provides a power supply and clock signal to 
$\PARTY{T}$;
direct access to the 
clock signal
means it is plausible 
for the former to influence the latter, e.g., causing a malfunction by supplying an irregular clock signal (or ``glitch'').

% -----------------------------------------------------------------------------

\levelstay{Material}

\leveldown{\lstinline[language={bash}]|\$\{ARCHIVE\}/CONF(ARCHIVE_PATH,CID)/\$\{USER\}.T|}

\DESCMATERIALT{%
\item $\FAULT$
      a  fault specification
      (represented as a                       decimal integer string, the format and semantics of which are explained further below),
\item $\PUB{m}$,
      a message
      (represented as a                   hexadecimal integer string),
}{
\item $\LEAK$,
      an execution latency measured in clock cycles
      (represented as a                       decimal integer string),
      and
\item $\PUB{\SIG}$,
      an RSA signature on $\PUB{m}$
      (represented as a                   hexadecimal integer string),
}%
Note that:

\begin{itemize}
\item Internally, 
      $\PARTY{T}$ 
      uses a co-processor to support multi-precision integer arithmetic: by
      storing operands in an internal, special-purpose register file, it is
      able to perform single-cycle computation of modular
           inversion (i.e., $1 / x     \pmod{n}$),
            negation (i.e., $  - x     \pmod{n}$),
            addition (i.e., $x +     y \pmod{n}$),
         subtraction (i.e., $x -     y \pmod{n}$),
      and
      multiplication (i.e., $x \cdot y \pmod{n}$).
\item $\PARTY{T}$ 
      uses the co-processor to realise a CRT-based~\cite{SCALE:QuiCou:82} 
      implementation of RSA decryption, using a Gauss-based recombination 
      step.
\item $\PARTY{E}$
      can induce a fault, via a ``glitch'' in the clock signal supplied to
      $\PARTY{T}$;
      the fault is controlled using $\FAULT$, in the sense that
      \[
      \begin{array}{c@{\;}c@{\;}c c c@{\;}l}
      \FAULT &<   & 0 &\Rightarrow& \mbox{no} & \mbox{fault induced                        } \\
      \FAULT &\geq& 0 &\Rightarrow&           & \mbox{fault induced in clock cycle $\FAULT$} \\
      \end{array}
      \]
      Through analysis of the fault model, it is known that inducing a fault 
      will impact the co-processor ALU; if a fault is induced in clock cycle
      $\FAULT$, computation by (or output of) the ALU in that clock cycle is
      corrupted (or randomised).
\end{itemize}

\levelstay{\lstinline[language={bash}]|\$\{ARCHIVE\}/CONF(ARCHIVE_PATH,CID)/\$\{USER\}.conf|}

\DESCMATERIALCONF{%
\item $\PUB{\CHAL{N}}$,     
      an RSA        modulus
      (represented as a                   hexadecimal integer string),
      and
\item $\PUB{\CHAL{e}}$,
      an RSA        public exponent
      (represented as a                   hexadecimal integer string),
      such that $\PUB{\CHAL{e}} \cdot \PRI{\CHAL{d}} \equiv 1 \pmod{\PRI{\Phi{(\CHAL{N})}}}$,
}%
Note that
$
\TUPLE{ \PUB{\CHAL{N}}, \PUB{\CHAL{e}} }
$
is
a     (known) RSA public  key 
associated with 
$
\TUPLE{ \PUB{\CHAL{N}}, \PRI{\CHAL{d}} } ,
$
i.e.,
the (unknown) RSA private key.

% -----------------------------------------------------------------------------

\levelup{Tasks}

\begin{enumerate}
\item \DESCTASKIMPL
      {$\PRI{\CHAL{d}}$}
      {\mbox{\lstinline[language={bash}]|./attack \$\{USER\}.T \$\{USER\}.conf|}}
\item \DESCTASKEXAM
      {\mbox{\lstinline[language={bash}]|\$\{ARCHIVE\}/CONF(ARCHIVE_PATH,CID)/\$\{USER\}.exam|}}
\end{enumerate}

% =============================================================================
