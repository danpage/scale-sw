% Copyright (C) 2017 Daniel Page <dan@phoo.org>
%
% Use of this source code is restricted per the CC BY-SA license, a copy of
% which can be found via http://creativecommons.org (and should be included 
% as LICENSE.txt within the associated archive or repository).

% =============================================================================

\leveldown{Background}

% TODO

\begin{center}
% Copyright (C) 2017 Daniel Page <dan@phoo.org>
%
% Use of this source code is restricted per the CC BY-SA license, a copy of
% which can be found via http://creativecommons.org (and should be included 
% as LICENSE.txt within the associated archive or repository).

\begin{tikzpicture}

% =============================================================================

% configure
 \tikzset{node distance={6cm},party/.style={rectangle,minimum width={2.0cm},minimum height={2.0cm}}}

% adversary and target device
 \node [                party] (E) {$\PARTY{E}$} ;
 \node [right of=E,draw,party] (T) {$\PARTY{T}$} ;

% embedded material
 \node at (T.south east) [anchor=south east,draw,rectangle] {\tiny $\PRI{\CHAL{k}}$} ;

%   direct input and output
 \draw [       >=stealth,->] (E.30)    to              
                             node [above]             {$\PUB{m}$} 
                             (E.30    -| T.west)  ;
 \draw [       >=stealth,<-] (E.330)   to              
                             node [below]             {$\PUB{c}$} 
                             (E.330   -| T.west)  ;

% indirect input and output
 \draw [dashed,>=stealth,<-] (T.north) to [bend right] 
                             node [above]             {$\FAULT \models \mbox{clock glitch}$}
                             (T.north -| E.north) ;
%\draw [dashed,>=stealth,->] (T.south) to [bend left]  
%                            node [below]             {} 
%                            (T.south -| E.south) ;

% computation
 \draw [       >=stealth,->] (T.30)    to [bend left,in=90,out=90,looseness=2] 
                             node [right,anchor=west] {$\PUB{c} = \SCOPE{\ID{AES-128}}{\ALG{Enc}}( \PRI{\CHAL{k}}, \PUB{m} )$}
                             (T.330) ;

% =============================================================================

\end{tikzpicture}


\end{center}

% TODO

% -----------------------------------------------------------------------------

\levelstay{Material}

\leveldown{\lstinline[language={bash}]|\$\{ARCHIVE\}/CONF(ARCHIVE_PATH,CID)/\$\{USER\}.D|}

This executable simulates the attack target $\PARTY{D}$.  When executed it 
 reads the following  input (one field per line)

\begin{itemize}
\item $\FAULT$
      a  fault specification
      (represented as a                       decimal integer string),
\item $\PUB{m}$,
      a message
      (represented as a                   hexadecimal integer string),
\end{itemize}

\noindent
from \lstinline[language={bash}]{stdin},
and 
writes the following output (one field per line)

\begin{itemize}
\item $\LEAK$,
      an execution time measured in clock cycles
      (represented as a                       decimal integer string),
      and
\item $\PUB{\SIG}$,
      an RSA signature on $m$
      (represented as a                   hexadecimal integer string),
\end{itemize}

\noindent
to   \lstinline[language={bash}]{stdout}.
Execution continues this way, i.e., repeatedly reading input then writing 
output, until $\PARTY{D}$ is forcibly terminated (or crashes).  
Note that:

\begin{itemize}
\item $\PARTY{D}$ uses a CRT-based implementation of RSA which is computed by
an external coprocessor which can perform basic arithmetic on large integers.
\item The fault specification $\FAULT$ induces a fault on this external
coprocessor.
\end{itemize}

\levelstay{\lstinline[language={bash}]|\$\{ARCHIVE\}/CONF(ARCHIVE_PATH,CID)/\$\{USER\}.conf|}

This file represents a set of attack parameters, with everything (e.g.,
all public values) $\PARTY{E}$ has access to by default.  It captures

\begin{itemize}
\item $\PUB{\CHAL{N}}$,     
      an RSA        modulus
      (represented as a                   hexadecimal integer string),
\item $\PUB{\CHAL{e}}$,
      an RSA        public exponent
      (represented as a                   hexadecimal integer string),
      st. $\PUB{\CHAL{e}} \cdot \PRI{\CHAL{d}} \equiv 1 \pmod{\PRI{\Phi{(\CHAL{N})}}}$,
\end{itemize}

\noindent
using one field per line.
More specifically, this represents the RSA public key 
$
\TUPLE{ \PUB{\CHAL{N}}, \PUB{\CHAL{e}} }
$
associated with the unknown RSA private key 
$
\TUPLE{ \PUB{\CHAL{N}}, \PRI{\CHAL{d}} } 
$
embedded in $\PARTY{D}$.

% -----------------------------------------------------------------------------

\levelup{Tasks}

\begin{enumerate}
\item Write a program that simulates the adversary $\PARTY{E}$ by attacking
      the simulated target, or, more specifically, that recovers the target 
      material $\PRI{\CHAL{d}}$.  
      When executed using a command of the form
      \[
      \mbox{\lstinline[language={bash}]|./attack \$\{USER\}.D \$\{USER\}.conf|}
      \]
      the attack should be invoked on the simulated target named (not some
      hard-coded alternative).  Use \lstinline[language={bash}]{stdout} to 
      print 
      a) any intermediate output you deem relevant, followed finally by 
      b) two lines which clearly detail the target material recovered plus
         the total number of interactions with the attack target.
\item Answer the exam-style questions in 
      \lstinline[language={bash}]|${ARCHIVE}/CONF(ARCHIVE_PATH,CID)/${USER}.exam|.
\end{enumerate}

% =============================================================================
