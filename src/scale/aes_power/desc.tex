% Copyright (C) 2017 Daniel Page <dan@phoo.org>
%
% Use of this source code is restricted per the CC BY-SA license, a copy of
% which can be found via http://creativecommons.org (and should be included 
% as LICENSE.txt within the associated archive or repository).

% =============================================================================

\leveldown{Background}

\IfStrEqCase{CONF(VERSION,CID)}{%
  {0}{Imagine you are tasked with attacking a device, denoted $\PARTY{T}$.
      The device is an ISO/IEC 7816 compliant contact-based smart-card: it
      houses an $8$-bit Intel $8051$ processor, and is typically used in a
      larger device as a co-processor module.  For example, it has support
      for secure key generation and storage, and off-load of cryptographic
      operations, such as AES, via a standardised protocol (or API).

      Provided physical access, $\PARTY{E}$ can interact with $\PARTY{T}$
      as follows:
  }%
  {1}{Imagine you are tasked with attacking a device, denoted $\PARTY{T}$.
      The device is an ISO/IEC 7816 compliant contact-based smart-card: it
      houses an $8$-bit Intel $8051$ processor, and is typically used in a
      larger device as a co-processor module.  For example, it has support
      for secure key generation and storage, and off-load of cryptographic
      operations, such as AES, via a standardised protocol (or API).  When
      granted physical access, $\PARTY{E}$ can interact with $\PARTY{T}$ 
      as follows:
  }%
  {2}{Imagine you are tasked with attacking a device, denoted $\PARTY{T}$.
      The device is a Self Encrypting Disk (SED): the underlying disk only
      ever stores encrypted data, which is encrypted (before writing) and
      decrypted (after reading) using XTS-AES~\cite{SCALE:IEEE:1619:07}.  
      The decryption of data uses an XTS-AES key embedded in $\PARTY{T}$, 
      which is initially derived from a user-selected password $\PRI{p}$, 
      i.e., 
      $
      \PRI{\CHAL{k}}              =  f( \PRI{p } )
      $ 
      for a derivation function $f$.  Each time the SED is powered-on, the
      password is {\em re}-entered as $\PRI{p'}$ and a check whether 
      $
      \PRI{\CHAL{k}} \stackrel{?}{=} f( \PRI{p'} )
      $ 
      then enforced; if the check fails, the SED refuses all interaction.
      $\PARTY{E}$ is lucky, and gets one-time access to $\PARTY{T}$ for a
      limited period of time while it is {\em already} powered-on (meaning 
      the password check has already been performed).  Put another way, 
      $\PARTY{E}$ can interact with $\PARTY{T}$ as follows:
  }%
}%

\begin{center}
% Copyright (C) 2017 Daniel Page <dan@phoo.org>
%
% Use of this source code is restricted per the CC BY-SA license, a copy of
% which can be found via http://creativecommons.org (and should be included 
% as LICENSE.txt within the associated archive or repository).

\begin{tikzpicture}

% =============================================================================

% configure
 \tikzset{node distance={6cm},party/.style={rectangle,minimum width={2.0cm},minimum height={2.0cm}}}

% adversary and target device
 \node [                party] (E) {$\PARTY{E}$} ;
 \node [right of=E,draw,party] (T) {$\PARTY{T}$} ;

% embedded material
 \node at (T.south east) [anchor=south east,draw,rectangle] {\tiny $\PRI{\CHAL{k}}$} ;

%   direct input and output
 \draw [       >=stealth,->] (E.30)    to              
                             node [above]             {$\PUB{m}$} 
                             (E.30    -| T.west)  ;
 \draw [       >=stealth,<-] (E.330)   to              
                             node [below]             {$\PUB{c}$} 
                             (E.330   -| T.west)  ;

% indirect input and output
 \draw [dashed,>=stealth,<-] (T.north) to [bend right] 
                             node [above]             {$\FAULT \models \mbox{clock glitch}$}
                             (T.north -| E.north) ;
%\draw [dashed,>=stealth,->] (T.south) to [bend left]  
%                            node [below]             {} 
%                            (T.south -| E.south) ;

% computation
 \draw [       >=stealth,->] (T.30)    to [bend left,in=90,out=90,looseness=2] 
                             node [right,anchor=west] {$\PUB{c} = \SCOPE{\ID{AES-128}}{\ALG{Enc}}( \PRI{\CHAL{k}}, \PUB{m} )$}
                             (T.330) ;

% =============================================================================

\end{tikzpicture}


\end{center}

\noindent
\IfStrEqCase{CONF(VERSION,CID)}{%
  {0}{That is, in each interaction $\PARTY{E}$ can (adaptively) send 
      a chosen AES  plaintext
      to $\PARTY{T}$; the device will
                encrypt that                              plaintext under the fixed, unknown     AES cipher key $\PRI{\CHAL{k}}$,
      and produce 
      the corresponding ciphertext.
  }%
  {1}{That is, in each interaction $\PARTY{E}$ can (adaptively) send 
      a chosen AES ciphertext
      to $\PARTY{T}$; the device will
                decrypt that                             ciphertext under the fixed, unknown     AES cipher key $\PRI{\CHAL{k}}$,
      and produce 
      the corresponding  plaintext.
  }%
  {2}{That is, in each interaction $\PARTY{E}$ can (adaptively) send 
      a block and sector address (the latter of which doubles as the XTS-AES tweak)
      to $\PARTY{T}$; the device will
      read then decrypt the selected but unknown XTS-AES ciphertext under the fixed, unknown XTS-AES        key $\PRI{\CHAL{k}}$ (per \cite[Section 5.4.1]{SCALE:IEEE:1619:07}),
      and produce 
      the corresponding  plaintext.
  }%
}%
Note that $\PARTY{E}$ supplies the power and clock signals to $\PARTY{T}$,
and can therefore measure the power consumed during each interaction: this
yields at least one sample per instruction executed due to the high sample 
rate of the oscilloscope used (relative to the clock frequency demanded by 
and supplied to $\PARTY{T}$).

% -----------------------------------------------------------------------------

\levelstay{Material}

\leveldown{\lstinline[language={bash}]|\$\{ARCHIVE\}/CONF(ARCHIVE_PATH,CID)/\$\{USER\}.T|}

This executable simulates the attack target $\PARTY{T}$.  When executed it 
 reads the following  input (one field per line)

\begin{itemize}
\IfStrEqCase{CONF(VERSION,CID)}{%
  {0}{\item $\PUB{m}$,
            a  ${1}$-block     AES  plaintext
            (represented as a  length-prefixed, hexadecimal octet   string),
  }%
  {1}{\item $\PUB{c}$,
            a  ${1}$-block     AES ciphertext
            (represented as a  length-prefixed, hexadecimal octet   string),
  }%
  {2}{
      \item $\PUB{j}$,
            a block  address 
            (represented as a                       decimal integer string),
            and            
      \item $\PUB{i}$,
            a sector address, i.e., 
            a  ${1}$-block XTS-AES tweak
            (represented as a  length-prefixed, hexadecimal octet   string),
  }%
}%
\end{itemize}

\noindent
from \lstinline[language={bash}]{stdin},
and 
writes the following output (one field per line)

\begin{itemize}
      \item $\LEAK$,
            a power consumption trace
            (whose representation is explained below),
            and
\IfStrEqCase{CONF(VERSION,CID)}{%
  {0}{\item $\PUB{c}$,
            a  ${1}$-block     AES ciphertext
            (represented as a  length-prefixed, hexadecimal octet   string),
  }%
  {1}{\item $\PUB{m}$,
            a  ${1}$-block     AES  plaintext
            (represented as a  length-prefixed, hexadecimal octet   string),
  }%
  {2}{\item $\PUB{m}$,
            a  ${1}$-block XTS-AES  plaintext
            (represented as a  length-prefixed, hexadecimal octet   string),
  }%
}%
\end{itemize}

\noindent
to   \lstinline[language={bash}]{stdout}.
Execution continues this way, i.e., repeatedly reading input then writing 
output, until $\PARTY{T}$ is forcibly terminated (or crashes).
Note that:

\begin{itemize}
\item $\PARTY{T}$ uses an AES-128 implementation,
      which clearly implies $128$-bit block and cipher key lengths;
      following the notation in~\cite[Figure 5]{SCALE:FIPS:197:01}, the 
      fact that 
      $
      Nb =  4
      $ 
      and 
      $
      Nr = 10
      $ 
      means a $( 4 \times 4 )$-element state matrix is used in a total of 
      $11$ rounds.
\item More concretely, it is an $8$-bit, memory-constrained implementation
      with the S-box held as a $\SI{256}{\byte}$ look-up table in memory. 
      In line with the goal of minimising the memory footprint, the round
      keys are not pre-computed: each encryption takes the cipher key and
      evolves it forward, step-by-step, to form successive round keys for
      use during key addition.  
\IfStrEq{CONF(VERSION,CID)}{1}{%
      However, decryption is more complicated because the round keys must 
      be used in the reverse order.  One possibility would be to store the
      last round key as well as the cipher key, and evolve this backward
      through each round key.  To avoid the increased demand on (secure,
      non-volatile) storage, $\PARTY{T}$ instead opts to first evolve the 
      cipher key forward into the last round key, {\em then} evolve this 
      backward through each round key.
}{}%
\IfStrEq{CONF(VERSION,CID)}{2}{%
      However, decryption is more complicated because the round keys must 
      be used in the reverse order.  One possibility would be to store the
      last round key as well as the cipher key, and evolve this backward
      through each round key.  To avoid the increased demand on (secure,
      non-volatile) storage, $\PARTY{T}$ instead opts to first evolve the 
      cipher key forward into the last round key, {\em then} evolve this 
      backward through each round key.
}{}%
\IfStrEq{CONF(VERSION,CID)}{2}{%
\item The underlying disk used by $\PARTY{T}$ has a $\SI{64}{\gibi\byte}$
      capacity and uses Advanced Format (AF) $\SI{4}{\kibi\byte}$ sectors.
      As such, each of the $16777216$ sectors contains $256$ blocks of AES 
      ciphertext: valid use of $\PARTY{T}$ therefore demands

      \begin{itemize}
      \item the block  address satisfies
            $
            0 \leq j <      256 ,
            $
            and
      \item the sector address satisfies
            $
            0 \leq i < 16777216 ,
            $
            which, since this implies only $\LSB[24](i)$ is relevant, is
            equivalent to saying $\MSB[104](i) = 0$.
      \end{itemize}
           
      \noindent
      $\PARTY{E}$ can use {\em any} $j$ and $i$ as input to $\PARTY{T}$: if 
      either is invalid, it uses a null ciphertext (i.e., $16$ zero bytes, 
      vs. a ciphertext read from the disk itself) but otherwise functions 
      as normal.
\item $\PARTY{T}$ pre-computes the $256$ possibilities of $\alpha^j$, which 
      are stored in a $\SI{4096}{\byte}$ look-up table.  As a result, it is 
      reasonable to assume AES invocations dominate the computation it will
      perform for each interaction.
}{}%
\item Each power consumption trace $\LEAK$ is a comma-separated line of the 
      form
      \[
      n, T_0, T_1, \ldots, T_{n-1}
      \]
      where
         
      \begin{itemize}
      \item $n$
            (represented as a                       decimal integer string),
            specifies the trace length, 
            and
      \item a given $T_i$ 
            (represented as a                       decimal integer string),
            specifies the $i$-th of $n$ power consumption samples, each 
            constituting an $8$-bit, unsigned decimal integer: in short, 
            this means $0 \leq T_i < 256$ for $0 \leq i < n$.
      \end{itemize}

\end{itemize}

\levelstay{\lstinline[language={bash}]|\$\{ARCHIVE\}/CONF(ARCHIVE_PATH,CID)/\$\{USER\}.R|}

In many side-channel and fault attacks, the attacker is able to perform an 
initial profiling or calibration phase: a common rationale for doing so is 
to support selection or fine-tuning of parameters for a subsequent attack.  
With this in mind, a simulated replica $\PARTY{R}$ of the attack target is 
also provided.
\IfStrEqCase{CONF(VERSION,CID)}{%
  {0}{$\PARTY{R}$ is identical to $\PARTY{T}$, bar the fact it
      reads the following  input (one field per line)
  }%
  {1}{$\PARTY{R}$ is identical to $\PARTY{T}$, bar the fact it
      reads the following  input (one field per line)
  }%
  {2}{In this case, $\PARTY{R}$ is a prototype version of $\PARTY{T}$ used
      to develop and debug the encryption mechanism: although identical in
      terms of power consumption, it 
      reads the following  input (one field per line)
  }%  
}%

\begin{itemize}
\IfStrEqCase{CONF(VERSION,CID)}{%
  {0}{\item $\PUB{\REPL{m}}$,
            a  ${1}$-block     AES  plaintext
            (represented as a  length-prefixed, hexadecimal octet   string),
            and            
      \item $\PUB{\REPL{k}}$,
            an                 AES cipher key
            (represented as a  length-prefixed, hexadecimal octet   string),
  }%
  {1}{\item $\PUB{\REPL{c}}$,
            a  ${1}$-block     AES ciphertext
            (represented as a  length-prefixed, hexadecimal octet   string),
            and            
      \item $\PUB{\REPL{k}}$,
            an                 AES cipher key
            (represented as a  length-prefixed, hexadecimal octet   string),
  }%
  {2}{\item $\PUB{\REPL{j}}$,
            a block  address 
            (represented as a                       decimal integer string),
      \item $\PUB{\REPL{i}}$,
            a sector address, i.e., 
            a  ${1}$-block XTS-AES tweak
            (represented as a  length-prefixed, hexadecimal octet   string),
      \item $\PUB{\REPL{c}}$,
            a  ${1}$-block XTS-AES ciphertext
            (represented as a  length-prefixed, hexadecimal octet   string),
            and            
      \item $\PUB{\REPL{k}}$,
            an             XTS-AES        key
            (represented as a  length-prefixed, hexadecimal octet   string),
  }%  
}%
\end{itemize}

\noindent
from \lstinline[language={bash}]{stdin}:
in contrast to $\PARTY{T}$, this means $\PARTY{R}$ uses the
\IfStrEqCase{CONF(VERSION,CID)}{%
  {0}{{\em chosen} cipher key $\PUB{\REPL{k}}$ 
      to encrypt the
           chosen  plaintext  $\PUB{\REPL{m}}$.
  }%
  {1}{{\em chosen} cipher key $\PUB{\REPL{k}}$ 
      to decrypt the
           chosen ciphertext  $\PUB{\REPL{c}}$.
  }%
  {2}{{\em chosen}        key $\PUB{\REPL{k}}$ 
      to decrypt the
      {\em chosen} ciphertext $\PUB{\REPL{c}}$.
  }%  
}%

% -----------------------------------------------------------------------------

\levelup{Tasks}

\begin{enumerate}
\item Write a program that simulates the adversary $\PARTY{E}$ by attacking
      the simulated target, or, more specifically, that recovers the target 
      material $\PRI{\CHAL{k}}$.  
      When executed using a command of the form
      \[
      \mbox{\lstinline[language={bash}]|./attack \$\{USER\}.T|}
      \]
      the attack should be invoked on the simulated target named (not some
      hard-coded alternative).  Use \lstinline[language={bash}]{stdout} to 
      print 
      a) any intermediate output you deem relevant, followed finally by 
      b) two lines which clearly detail the target material recovered plus
         the total number of interactions with the attack target.
\item Answer the exam-style questions in 
      \lstinline[language={bash}]|${ARCHIVE}/CONF(ARCHIVE_PATH,CID)/${USER}.exam|.
\end{enumerate}

% =============================================================================
