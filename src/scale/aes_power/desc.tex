% Copyright (C) 2017 Daniel Page <dan@phoo.org>
%
% Use of this source code is restricted per the CC BY-SA license, a copy of
% which can be found via http://creativecommons.org (and should be included 
% as LICENSE.txt within the associated archive or repository).

% =============================================================================

\leveldown{Background}

\IfStrEqCase{CONF(VERSION,CID)}{%
  {0}{Imagine you are tasked with attacking a device, denoted 
      $\PARTY{T}$, 
      which houses an $8$-bit Intel $8051$ micro-processor.  More specifically,
      $\PARTY{T}$ 
      represents an ISO/IEC 7816 compliant contact-based smart-card used within 
      larger devices as a cryptographic co-processor module: 
      using a standardised protocol (or API), it supports secure key generation 
      and storage plus off-load of some cryptographic operations.
      \DESCINTRO[leak]
                {% Copyright (C) 2017 Daniel Page <dan@phoo.org>
%
% Use of this source code is restricted per the CC BY-SA license, a copy of
% which can be found via http://creativecommons.org (and should be included 
% as LICENSE.txt within the associated archive or repository).

\begin{tikzpicture}

% =============================================================================

% configure
 \tikzset{node distance={6cm},party/.style={rectangle,minimum width={2.0cm},minimum height={2.0cm}}}

% adversary and target device
 \node [                party] (E) {$\PARTY{E}$} ;
 \node [right of=E,draw,party] (T) {$\PARTY{T}$} ;

% embedded material
 \node at (T.south east) [anchor=south east,draw,rectangle] {\tiny $\PRI{\CHAL{k}}$} ;

%   direct input and output
 \draw [       >=stealth,->] (E.30)    to              
                             node [above]             {$\PUB{m}$} 
                             (E.30    -| T.west)  ;
 \draw [       >=stealth,<-] (E.330)   to              
                             node [below]             {$\PUB{c}$} 
                             (E.330   -| T.west)  ;

% indirect input and output
 \draw [dashed,>=stealth,<-] (T.north) to [bend right] 
                             node [above]             {$\FAULT \models \mbox{clock glitch}$}
                             (T.north -| E.north) ;
%\draw [dashed,>=stealth,->] (T.south) to [bend left]  
%                            node [below]             {} 
%                            (T.south -| E.south) ;

% computation
 \draw [       >=stealth,->] (T.30)    to [bend left,in=90,out=90,looseness=2] 
                             node [right,anchor=west] {$\PUB{c} = \SCOPE{\ID{AES-128}}{\ALG{Enc}}( \PRI{\CHAL{k}}, \PUB{m} )$}
                             (T.330) ;

% =============================================================================

\end{tikzpicture}

}
                {a cipher key}
                {a  plaintext}
                {i.e., it encrypts the  plaintext using the cipher key}
                {a ciphertext}
  }%
  {1}{Imagine you are tasked with attacking a device, denoted 
      $\PARTY{T}$, 
      which houses an $8$-bit Intel $8051$ micro-processor.  More specifically,
      $\PARTY{T}$ 
      represents an ISO/IEC 7816 compliant contact-based smart-card used within 
      larger devices as a cryptographic co-processor module: 
      using a standardised protocol (or API), it supports secure key generation 
      and storage plus off-load of some cryptographic operations.
      \DESCINTRO[leak]
                {% Copyright (C) 2017 Daniel Page <dan@phoo.org>
%
% Use of this source code is restricted per the CC BY-SA license, a copy of
% which can be found via http://creativecommons.org (and should be included 
% as LICENSE.txt within the associated archive or repository).

\begin{tikzpicture}

% =============================================================================

% configure
 \tikzset{node distance={6cm},party/.style={rectangle,minimum width={2.0cm},minimum height={2.0cm}}}

% adversary and target device
 \node [                party] (E) {$\PARTY{E}$} ;
 \node [right of=E,draw,party] (T) {$\PARTY{T}$} ;

% embedded material
 \node at (T.south east) [anchor=south east,draw,rectangle] {\tiny $\PRI{\CHAL{k}}$} ;

%   direct input and output
 \draw [       >=stealth,->] (E.30)    to              
                             node [above]             {$\PUB{m}$} 
                             (E.30    -| T.west)  ;
 \draw [       >=stealth,<-] (E.330)   to              
                             node [below]             {$\PUB{c}$} 
                             (E.330   -| T.west)  ;

% indirect input and output
 \draw [dashed,>=stealth,<-] (T.north) to [bend right] 
                             node [above]             {$\FAULT \models \mbox{clock glitch}$}
                             (T.north -| E.north) ;
%\draw [dashed,>=stealth,->] (T.south) to [bend left]  
%                            node [below]             {} 
%                            (T.south -| E.south) ;

% computation
 \draw [       >=stealth,->] (T.30)    to [bend left,in=90,out=90,looseness=2] 
                             node [right,anchor=west] {$\PUB{c} = \SCOPE{\ID{AES-128}}{\ALG{Enc}}( \PRI{\CHAL{k}}, \PUB{m} )$}
                             (T.330) ;

% =============================================================================

\end{tikzpicture}

}
                {a cipher key}
                {a ciphertext}
                {i.e., it decrypts the ciphertext using the cipher key}
                {a  plaintext}
  }%
  {2}{Imagine you are tasked with attacking a device, denoted 
      $\PARTY{T}$.
      The device is a Self Encrypting Disk (SED): the underlying disk only
      ever stores encrypted data, which is encrypted (before writing) and
      decrypted (after reading) using XTS-AES~\cite{SCALE:IEEE:1619:07}.  
      The decryption of data uses an XTS-AES key embedded in 
      $\PARTY{T}$, 
      which is initially derived from a user-selected password $\PRI{p}$, 
      i.e., 
      $
      \PRI{\CHAL{k}}              =  f( \PRI{p} )
      $ 
      for a derivation function $f$.  
      Each time the SED is powered-on, a password attempt/guess
      $
      \PRI{g}
      $ 
      is entered and a check whether 
      $
      \PRI{\CHAL{k}} \stackrel{?}{=} f( \PRI{g} )
      $ 
      then enforced; if the check fails, then the SED will refuses all further
      attempts to interact with it.
      $\PARTY{E}$ 
      is lucky, and gets one-time access to 
      $\PARTY{T}$ 
      for a limited period of time while it is already powered-on (implying the
      the password check has already been performed).
      \DESCINTRO[leak]
                {% Copyright (C) 2017 Daniel Page <dan@phoo.org>
%
% Use of this source code is restricted per the CC BY-SA license, a copy of
% which can be found via http://creativecommons.org (and should be included 
% as LICENSE.txt within the associated archive or repository).

\begin{tikzpicture}

% =============================================================================

% configure
 \tikzset{node distance={6cm},party/.style={rectangle,minimum width={2.0cm},minimum height={2.0cm}}}

% adversary and target device
 \node [                party] (E) {$\PARTY{E}$} ;
 \node [right of=E,draw,party] (T) {$\PARTY{T}$} ;

% embedded material
 \node at (T.south east) [anchor=south east,draw,rectangle] {\tiny $\PRI{\CHAL{k}}$} ;

%   direct input and output
 \draw [       >=stealth,->] (E.30)    to              
                             node [above]             {$\PUB{m}$} 
                             (E.30    -| T.west)  ;
 \draw [       >=stealth,<-] (E.330)   to              
                             node [below]             {$\PUB{c}$} 
                             (E.330   -| T.west)  ;

% indirect input and output
 \draw [dashed,>=stealth,<-] (T.north) to [bend right] 
                             node [above]             {$\FAULT \models \mbox{clock glitch}$}
                             (T.north -| E.north) ;
%\draw [dashed,>=stealth,->] (T.south) to [bend left]  
%                            node [below]             {} 
%                            (T.south -| E.south) ;

% computation
 \draw [       >=stealth,->] (T.30)    to [bend left,in=90,out=90,looseness=2] 
                             node [right,anchor=west] {$\PUB{c} = \SCOPE{\ID{AES-128}}{\ALG{Enc}}( \PRI{\CHAL{k}}, \PUB{m} )$}
                             (T.330) ;

% =============================================================================

\end{tikzpicture}

}
                {some key material}
                {a block and sector address}
                {i.e., it reads and decrypts the addressed ciphertext block}
                {a  plaintext}
  }%
}%
Note that
$\PARTY{E}$ 
provides a power supply and clock signal to 
$\PARTY{T}$;
direct access to the 
power supply 
means it is plausible 
for the former to observe power consumption of the later.
Doing so will yield at least one sample per instruction executed, due to 
the sample rate of the oscilloscope used relative to the clock frequency 
demanded by and supplied to 
$\PARTY{T}$.

% -----------------------------------------------------------------------------

\levelstay{Material}

\leveldown{\lstinline[language={bash}]|\$\{ARCHIVE\}/CONF(ARCHIVE_PATH,CID)/\$\{USER\}.T|}

\IfStrEqCase{CONF(VERSION,CID)}{%
  {0}{\DESCMATERIALT{%
        \item $\PUB{m}$,
              a  ${1}$-block     AES  plaintext
              (represented as a  length-prefixed, hexadecimal octet   string),
      }{%
        \item $\LEAK$,
              a power consumption trace
              (whose representation is explained below),
              and
        \item $\PUB{c}$,
              a  ${1}$-block     AES ciphertext
              (represented as a  length-prefixed, hexadecimal octet   string),
      }%
  }%
  {1}{\DESCMATERIALT{%
        \item $\PUB{c}$,
              a  ${1}$-block     AES ciphertext
              (represented as a  length-prefixed, hexadecimal octet   string),
      }{%
        \item $\LEAK$,
              a power consumption trace
              (whose representation is explained below),
              and
        \item $\PUB{m}$,
              a  ${1}$-block     AES  plaintext
              (represented as a  length-prefixed, hexadecimal octet   string),
      }%
  }%
  {2}{\DESCMATERIALT{%
        \item $\PUB{j}$,
              a block  address 
              (represented as a                       decimal integer string),
              and            
        \item $\PUB{i}$,
              a sector address, i.e., 
              a  ${1}$-block XTS-AES tweak
              (represented as a  length-prefixed, hexadecimal octet   string),
      }{%
        \item $\LEAK$,
              a power consumption trace
              (whose representation is explained below),
              and
        \item $\PUB{m}$,
              a  ${1}$-block XTS-AES  plaintext
              (represented as a  length-prefixed, hexadecimal octet   string),
      }%
  }%
}%
Note that:

\begin{itemize}
\item $\PARTY{T}$ 
      uses an AES-128 implementation, which implies $128$-bit block and 
      cipher key lengths;
      following the notation in~\cite[Figure 5]{SCALE:FIPS:197:01}, note
      that 
      $
      Nb =  4
      $ 
      and 
      $
      Nr = 10
      $ 
      means a $( 4 \times 4 )$-element state matrix is used in a total of 
      $11$ rounds.
\item More concretely, it is an $8$-bit, memory-constrained implementation
      with the S-box held as a $\SI{256}{\byte}$ look-up table in memory. 
      In line with the goal of minimising the memory footprint, the round
      keys are not pre-computed: each encryption takes the cipher key and
      evolves it forward, step-by-step, to form successive round keys for
      use during key addition.  
\IfStrEq{CONF(VERSION,CID)}{1}{%
      However, decryption is more complicated because the round keys must 
      be used in the reverse order.  One possibility would be to store the
      last round key as well as the cipher key, and evolve this backward
      through each round key.  To avoid the increased demand on (secure,
      non-volatile) storage, $\PARTY{T}$ instead opts to first evolve the 
      cipher key forward into the last round key, {\em then} evolve this 
      backward through each round key.
}{}%
\IfStrEq{CONF(VERSION,CID)}{2}{%
      However, decryption is more complicated because the round keys must 
      be used in the reverse order.  One possibility would be to store the
      last round key as well as the cipher key, and evolve this backward
      through each round key.  To avoid the increased demand on (secure,
      non-volatile) storage, $\PARTY{T}$ instead opts to first evolve the 
      cipher key forward into the last round key, {\em then} evolve this 
      backward through each round key.
}{}%
\IfStrEq{CONF(VERSION,CID)}{2}{%
\item The underlying disk used by $\PARTY{T}$ has a 
      $\SI{64}{\gibi\byte}$
      capacity and uses Advanced Format (AF) 
      $\SI {4}{\kibi\byte}$ 
      sectors.
      As such, each of the $16777216$ sectors contains $256$ blocks of AES 
      ciphertext: valid use of $\PARTY{T}$ therefore demands

      \begin{itemize}
      \item the block  address satisfies
            $
            0 \leq j <      256 ,
            $
            and
      \item the sector address satisfies
            $
            0 \leq i < 16777216 ,
            $
            which, since this implies only $\LSB[24](i)$ is relevant, is
            equivalent to saying $\MSB[104](i) = 0$.
      \end{itemize}
           
      \noindent
      $\PARTY{E}$ can use {\em any} $j$ and $i$ as input to $\PARTY{T}$: if 
      either is invalid, it uses a null ciphertext (i.e., $16$ zero bytes, 
      vs. a ciphertext read from the disk itself) but otherwise functions 
      as normal.
\item $\PARTY{T}$ pre-computes the $256$ possibilities of $\alpha^j$, which 
      are stored in a $\SI{4096}{\byte}$ look-up table.  As a result, it is 
      reasonable to assume AES invocations dominate the computation it will
      perform for each interaction.
}{}%
\item Each power consumption trace $\LEAK$ is a comma-separated line of the 
      form
      \[
      l, \lambda_0, \lambda_1, \ldots, \lambda_{l-1}
      \]
      where
         
      \begin{itemize}
      \item $l$
            (represented as a                       decimal integer string),
            specifies the trace length, 
            and
      \item a given $\lambda_i$ 
            (represented as a                       decimal integer string),
            specifies the $i$-th of $l$ power consumption samples, each 
            constituting an $8$-bit, unsigned decimal integer: in short, 
            this means $0 \leq \lambda_i < 256$ for $0 \leq i < l$.
      \end{itemize}

\end{itemize}

\levelstay{\lstinline[language={bash}]|\$\{ARCHIVE\}/CONF(ARCHIVE_PATH,CID)/\$\{USER\}.R|}

\DESCMATERIALR{%
\IfStrEqCase{CONF(VERSION,CID)}{%
  {0}{\item $\PUB{\REPL{m}}$,
            a  ${1}$-block     AES  plaintext
            (represented as a  length-prefixed, hexadecimal octet   string),
            and            
      \item $\PUB{\REPL{k}}$,
            an                 AES cipher key
            (represented as a  length-prefixed, hexadecimal octet   string),
  }%
  {1}{\item $\PUB{\REPL{c}}$,
            a  ${1}$-block     AES ciphertext
            (represented as a  length-prefixed, hexadecimal octet   string),
            and            
      \item $\PUB{\REPL{k}}$,
            an                 AES cipher key
            (represented as a  length-prefixed, hexadecimal octet   string),
  }%
  {2}{\item $\PUB{\REPL{j}}$,
            a block  address 
            (represented as a                       decimal integer string),
      \item $\PUB{\REPL{i}}$,
            a sector address, i.e., 
            a  ${1}$-block XTS-AES tweak
            (represented as a  length-prefixed, hexadecimal octet   string),
      \item $\PUB{\REPL{c}}$,
            a  ${1}$-block XTS-AES ciphertext
            (represented as a  length-prefixed, hexadecimal octet   string),
            and            
      \item $\PUB{\REPL{k}}$,
            an             XTS-AES        key
            (represented as a  length-prefixed, hexadecimal octet   string),
  }%  
}%
}%
In contrast to 
$\PARTY{T}$, 
this means 
$\PARTY{R}$ 
will use the
\IfStrEqCase{CONF(VERSION,CID)}{%
  {0}{{\em chosen} cipher key $\PUB{\REPL{k}}$ 
      to encrypt the
           chosen  plaintext  $\PUB{\REPL{m}}$.
  }%
  {1}{{\em chosen} cipher key $\PUB{\REPL{k}}$ 
      to decrypt the
           chosen ciphertext  $\PUB{\REPL{c}}$.
  }%
  {2}{{\em chosen}        key $\PUB{\REPL{k}}$ 
      to decrypt the
      {\em chosen} ciphertext $\PUB{\REPL{c}}$.
  }%  
}%

% -----------------------------------------------------------------------------

\levelup{Tasks}

\begin{enumerate}
\item \DESCTASKIMPL
      {$\PRI{\CHAL{k}}$}
      {\mbox{\lstinline[language={bash}]|./attack \$\{USER\}.T|}}
\item \DESCTASKEXAM
      {\mbox{\lstinline[language={bash}]|\$\{ARCHIVE\}/CONF(ARCHIVE_PATH,CID)/\$\{USER\}.exam|}}
\end{enumerate}

% =============================================================================
