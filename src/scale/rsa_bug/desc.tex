% Copyright (C) 2017 Daniel Page <dan@phoo.org>
%
% Use of this source code is restricted per the CC BY-SA license, a copy of
% which can be found via http://creativecommons.org (and should be included 
% as LICENSE.txt within the associated archive or repository).

% =============================================================================

\leveldown{Background}

Imagine you are tasked with attacking a server, denoted 
$\PARTY{T}$, 
which houses a ``cloned'' $64$-bit MIPS64 micro-processor: the reduced cost 
(versus a legitimate alternative) 
made it an attractive option for the owner, who was forced to economise when
upgrading an old data center.
$\PARTY{T}$ 
is used by several front-line e-commerce servers, which offload computation 
relating to the TLS handshake.  More specifically, 
$\PARTY{T}$ 
is used to compute the RSA decryption required by RSA-based key exchange.
\DESCINTRO[none]
{% Copyright (C) 2017 Daniel Page <dan@phoo.org>
%
% Use of this source code is restricted per the CC BY-SA license, a copy of
% which can be found via http://creativecommons.org (and should be included 
% as LICENSE.txt within the associated archive or repository).

\begin{tikzpicture}

% =============================================================================

% configure
 \tikzset{node distance={6cm},party/.style={rectangle,minimum width={2.0cm},minimum height={2.0cm}}}

% adversary and target device
 \node [                party] (E) {$\PARTY{E}$} ;
 \node [right of=E,draw,party] (T) {$\PARTY{T}$} ;

% embedded material
 \node at (T.south east) [anchor=south east,draw,rectangle] {\tiny $\PRI{\CHAL{k}}$} ;

%   direct input and output
 \draw [       >=stealth,->] (E.30)    to              
                             node [above]             {$\PUB{m}$} 
                             (E.30    -| T.west)  ;
 \draw [       >=stealth,<-] (E.330)   to              
                             node [below]             {$\PUB{c}$} 
                             (E.330   -| T.west)  ;

% indirect input and output
 \draw [dashed,>=stealth,<-] (T.north) to [bend right] 
                             node [above]             {$\FAULT \models \mbox{clock glitch}$}
                             (T.north -| E.north) ;
%\draw [dashed,>=stealth,->] (T.south) to [bend left]  
%                            node [below]             {} 
%                            (T.south -| E.south) ;

% computation
 \draw [       >=stealth,->] (T.30)    to [bend left,in=90,out=90,looseness=2] 
                             node [right,anchor=west] {$\PUB{c} = \SCOPE{\ID{AES-128}}{\ALG{Enc}}( \PRI{\CHAL{k}}, \PUB{m} )$}
                             (T.330) ;

% =============================================================================

\end{tikzpicture}

}
{a private key}
{a ciphertext}
{i.e., it decrypts the ciphertext using the private key}
{a  plaintext}
Unfortunately, however, the micro-processor used by 
$\PARTY{T}$ 
has a defect: hardware within the ALU which supports $( 64 \times 64 )$-bit 
integer multiplication has a (deterministic) bug, which means
the    output (i.e., the product $x \cdot y$)
is incorrect for 
certain input (i.e., certain $x$ and/or $y$).

% -----------------------------------------------------------------------------

\levelstay{Material}

\leveldown{\lstinline[language={bash}]|\$\{ARCHIVE\}/CONF(ARCHIVE_PATH,CID)/\$\{USER\}.T|}

\DESCMATERIALT{%
\item $\PUB{c}$,
      an RSA ciphertext
      (represented as a                   hexadecimal integer string),
}{%
\item $\PUB{m}$,
      an RSA plaintext
      (represented as a                   hexadecimal integer string),
}%
Note that:

\begin{itemize}
\item Because $\PARTY{T}$ houses a $64$-bit micro-processor, it employs a
      base-$2^{64}$ representation of multi-precision integers throughout.  
      Put another way, since $w = 64$, it selects $b = 2^{w} = 2^{64}$.
\item Rather than a CRT-based approach, $\PARTY{T}$ uses a left-to-right 
      binary exponentiation~\cite[Section 2.1]{SCALE:Gordon:85} algorithm
      to compute 
      $
      \PUB{m} = \PUB{c}^{\PRI{\CHAL{d}}} \pmod{\PUB{\CHAL{N}}} .
      $
      Montgomery multiplication~\cite{SCALE:Montgomery:85} is harnessed to
      improve efficiency; following notation in~\cite{SCALE:KocAcaKal:96}, 
      a CIOS-based~\cite[Section 5]{SCALE:KocAcaKal:96} approach is used
      to integrate\footnote{%
      \ALG{MonPro}~\cite[Section 2]{SCALE:KocAcaKal:96} perhaps offers a 
      more direct way to explain this: steps $1$ and $2$ are the integer 
      multiplication and Montgomery reduction written both separately and 
      abstractly, whereas $\PARTY{T}$ realises them concretely using the 
      integrated CIOS algorithm.
      } an integer multiplication with a subsequent Montgomery reduction.
\item Following the notation in~\cite{SCALE:Gordon:85}, 
      $
      \PRI{\CHAL{d}}
      $ 
      is assumed to have $l + 1$ bits such that
      $
      \PRI{\CHAL{d}_l} = 1 
      $
      for some $l$.  However, it has been (artificially) selected so
      $
      0 \leq \PRI{\CHAL{d}} < 2^{CONF(LOG_D_MAX,CID)}
      $
      to limit the duration of an attack: you should not rely on this fact, 
      implying your attack could succeed for {\em any} 
      $
      \PRI{\CHAL{d}}
      $ 
      if afforded enough time.
\item Consider the base-$2^{CONF(POISONED_N,CID)}$ expansion of an integer 
      $t$, i.e.,
      $
      t = \RADIX{\LIST{ t_0, t_1, \ldots, t_{n-1} }}{2^{CONF(POISONED_N,CID)}} .
      $
      We say that $t$ is $\gamma$-poisoned iff. there exists an $i$ such
      that $0 \leq i < n$ and $t_i = \gamma$, i.e., at least one of the 
      limbs is equal to $\gamma$. 
      Recall that the integer multiplier used by $\PARTY{T}$ is defective: 
      for
      \[
      \begin{array}{lcl}
      \alpha &=& \RADIX{CONF(POISONED_X,CID)}{CONF(POISONED_N,CID)} \\
      \beta  &=& \RADIX{CONF(POISONED_Y,CID)}{CONF(POISONED_N,CID)} \\
      \end{array}
      \]
      it will incorrectly compute the product $x \cdot y$ iff. 
      $x$ is $\alpha$-poisoned 
      and 
      $y$ is  $\beta$-poisoned.
\end{itemize}

\levelstay{\lstinline[language={bash}]|\$\{ARCHIVE\}/CONF(ARCHIVE_PATH,CID)/\$\{USER\}.conf|}

\DESCMATERIALCONF{%
\item $\PUB{\CHAL{N}}$,     
      an RSA        modulus
      (represented as a                   hexadecimal integer string),
      and
\item $\PUB{\CHAL{e}}$,
      an RSA        public exponent
      (represented as a                   hexadecimal integer string),
      such that $\PUB{\CHAL{e}} \cdot \PRI{\CHAL{d}} \equiv 1 \pmod{\PRI{\Phi{(\CHAL{N})}}}$,
}%
Note that
$
\TUPLE{ \PUB{\CHAL{N}}, \PUB{\CHAL{e}} }
$
is
a     (known) RSA public  key 
associated with 
$
\TUPLE{ \PUB{\CHAL{N}}, \PRI{\CHAL{d}} } ,
$
i.e.,
the (unknown) RSA private key.

% -----------------------------------------------------------------------------

\levelup{Tasks}

\begin{enumerate}
\item \DESCTASKIMPL
      {$\PRI{\CHAL{d}}$}
      {\mbox{\lstinline[language={bash}]|./attack \$\{USER\}.T \$\{USER\}.conf|}}
\item \DESCTASKEXAM
      {\mbox{\lstinline[language={bash}]|\$\{ARCHIVE\}/CONF(ARCHIVE_PATH,CID)/\$\{USER\}.exam|}}
\end{enumerate}

% =============================================================================
