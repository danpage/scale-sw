% Copyright (C) 2017 Daniel Page <dan@phoo.org>
%
% Use of this source code is restricted per the CC BY-SA license, a copy of
% which can be found via http://creativecommons.org (and should be included 
% as LICENSE.txt within the associated archive or repository).

% =============================================================================

\leveldown{Background}

Imagine you are tasked with attacking a server, denoted $\PARTY{T}$, 
which houses a ``cloned'' $64$-bit MIPS64 processor: the low cost (versus a 
legitimate alternative) made it an attractive option for the owner, who was
forced to economise when upgrading an old data center.
$\PARTY{T}$ is used by several front-line e-commerce servers, which offload
computation relating to the TLS handshake.  More specifically, $\PARTY{T}$ 
is used to compute an RSA decryption (in software, of the pre-master secret)
whenever RSA-based key exchange is required.
By leveraging access to the network, 
an attacker $\PARTY{E}$ can interact with $\PARTY{T}$ as follows:

\begin{center}
% Copyright (C) 2017 Daniel Page <dan@phoo.org>
%
% Use of this source code is restricted per the CC BY-SA license, a copy of
% which can be found via http://creativecommons.org (and should be included 
% as LICENSE.txt within the associated archive or repository).

\begin{tikzpicture}

% =============================================================================

% configure
 \tikzset{node distance={6cm},party/.style={rectangle,minimum width={2.0cm},minimum height={2.0cm}}}

% adversary and target device
 \node [                party] (E) {$\PARTY{E}$} ;
 \node [right of=E,draw,party] (T) {$\PARTY{T}$} ;

% embedded material
 \node at (T.south east) [anchor=south east,draw,rectangle] {\tiny $\PRI{\CHAL{k}}$} ;

%   direct input and output
 \draw [       >=stealth,->] (E.30)    to              
                             node [above]             {$\PUB{m}$} 
                             (E.30    -| T.west)  ;
 \draw [       >=stealth,<-] (E.330)   to              
                             node [below]             {$\PUB{c}$} 
                             (E.330   -| T.west)  ;

% indirect input and output
%\draw [dashed,>=stealth,<-] (T.north) to [bend right] 
%                            node [above]             {} 
%                            (T.north -| E.north) ;
 \draw [dashed,>=stealth,->] (T.south) to [bend left]  
                             node [below]             {$\LEAK  \models \mbox{execution latency}$} 
                             (T.south -| E.south) ;

% computation
 \draw [       >=stealth,->] (T.30)    to [bend left,in=90,out=90,looseness=2] 
                             node [right,anchor=west] {$\PUB{c} = \SCOPE{\ID{DES}}{\ALG{Enc}}( \PRI{\CHAL{k}}, \PUB{m} )$}
                             (T.330) ;

% =============================================================================

\end{tikzpicture}

\end{center}

\noindent
That is, in each interaction $\PARTY{E}$ can (adaptively) send 
a chosen RSA ciphertext
to $\PARTY{T}$; the device will
decrypt that ciphertext under the fixed, unknown RSA private key $\TUPLE{ \PUB{\CHAL{N}}, \PRI{\CHAL{d}} }$ 
and produce 
the corresponding  plaintext.
Unfortunately, however, the processor used by $\PARTY{T}$ has an important
defect: the $( 64 \times 64 )$-bit integer multiplier implementation has a
(deterministic) bug, meaning the output is incorrect for specific inputs.

% -----------------------------------------------------------------------------

\levelstay{Material}

\leveldown{\lstinline[language={bash}]|\$\{ARCHIVE\}/CONF(ARCHIVE_PATH,CID)/\$\{USER\}.T|}

This executable simulates the attack target $\PARTY{T}$.  When executed it 
 reads the following  input

\begin{itemize}
\item $\PUB{c}$,
      an RSA ciphertext
      (represented as a                   hexadecimal integer string),
\end{itemize}

\noindent
from \lstinline[language={bash}]{stdin},
and 
writes the following output

\begin{itemize}
\item $\PUB{m}$,
      an RSA plaintext
      (represented as a                   hexadecimal integer string),
\end{itemize}

\noindent
to   \lstinline[language={bash}]{stdout} (using one field per line in both cases).  
Execution continues this way, i.e., repeatedly reading input then writing 
output, until $\PARTY{T}$ is forcibly terminated (or crashes).  
Note that:

\begin{itemize}
\item $\PARTY{T}$ houses a $64$-bit processor, so it uses a base-$2^{64}$ 
      representation of multi-precision integers throughout.  Put another 
      way, since $w = 64$, it selects $b = 2^{w} = 2^{64}$.
\item Rather than a CRT-based approach, $\PARTY{T}$ uses a left-to-right 
      binary exponentiation~\cite[Section 2.1]{SCALE:Gordon:85} algorithm
      to compute 
      $
      \PUB{m} = \PUB{c}^{\PRI{\CHAL{d}}} \pmod{\PUB{\CHAL{N}}} .
      $
      Montgomery multiplication~\cite{SCALE:Montgomery:85} is harnessed to
      improve efficiency; following notation in~\cite{SCALE:KocAcaKal:96}, 
      a CIOS-based~\cite[Section 5]{SCALE:KocAcaKal:96} approach is used
      to integrate\footnote{%
      \ALG{MonPro}~\cite[Section 2]{SCALE:KocAcaKal:96} perhaps offers a 
      more direct way to explain this: steps $1$ and $2$ are the integer 
      multiplication and Montgomery reduction written both separately and 
      abstractly, whereas $\PARTY{T}$ realises them concretely using the 
      integrated CIOS algorithm.
      } an integer multiplication with a subsequent Montgomery reduction.
\item Following the notation in~\cite{SCALE:Gordon:85}, $\PRI{\CHAL{d}}$
      is assumed to have $l+1$ bits st. $\PRI{\CHAL{d}_l} = 1$.  However, 
      it has been (artificially) selected st. 
      $
      0 \leq \PRI{\CHAL{d}} < 2^{CONF(LOG_D_MAX,CID)}
      $
      to limit the mount of time an attack requires: you should not rely 
      on this fact, implying your attack should succeed for {\em any}
      $\PRI{\CHAL{d}}$ given enough time.
\item Consider the base-$2^{CONF(POISONED_N,CID)}$ expansion of an integer 
      $t$, i.e.,
      \[
      t = \RADIX{\LIST{ t_0, t_1, \ldots, t_{n-1} }}{2^{CONF(POISONED_N,CID)}} .
      \]
      We say that $t$ is $\gamma$-poisoned iff. there exists an $i$ st.
      $0 \leq i < n$ and $t_i = \gamma$, i.e., at least one of the limbs 
      is equal to $\gamma$.  Recall that the integer multiplier used by
      $\PARTY{T}$ is defective: for
      \[
      \begin{array}{lcl}
      \alpha &=& \RADIX{CONF(POISONED_X,CID)}{CONF(POISONED_N,CID)} \\
      \beta  &=& \RADIX{CONF(POISONED_Y,CID)}{CONF(POISONED_N,CID)} \\
      \end{array}
      \]
      it will incorrectly compute the product $x \cdot y$ iff. $x$ is 
      $\alpha$-poisoned and $y$ is $\beta$-poisoned.
\end{itemize}

\levelstay{\lstinline[language={bash}]|\$\{ARCHIVE\}/CONF(ARCHIVE_PATH,CID)/\$\{USER\}.conf|}

This file represents a set of attack parameters, with everything (e.g.,
all public values) $\PARTY{E}$ has access to by default.  It contains 

\begin{itemize}
\item $\PUB{\CHAL{N}}$,
      an RSA modulus
      (represented as a                   hexadecimal integer string),
      and
\item $\PUB{\CHAL{e}}$,
      an RSA public exponent
      (represented as a                   hexadecimal integer string),
      st. $\PUB{\CHAL{e}} \cdot \PRI{\CHAL{d}} \equiv 1 \pmod{\PRI{\Phi(\CHAL{N})}}$,
\end{itemize}

\noindent
with one field per line.
More specifically, this represents the RSA public key 
$
\TUPLE{ \PUB{\CHAL{N}}, \PUB{\CHAL{e}} }
$
associated with the unknown RSA private key 
$
\TUPLE{ \PUB{\CHAL{N}}, \PRI{\CHAL{d}} }
$
embedded in $\PARTY{T}$.

% -----------------------------------------------------------------------------

\levelup{Tasks}

\begin{enumerate}
\item Write a program that simulates the adversary $\PARTY{E}$ by attacking
      the simulated target, or, more specifically, that recovers the target 
      material $\PRI{\CHAL{d}}$.  
      When executed using a command of the form
      \[
      \mbox{\lstinline[language={bash}]|./attack \$\{USER\}.T \$\{USER\}.conf|}
      \]
      the attack should be invoked on the simulated target named (not some
      hard-coded alternative).  Use \lstinline[language={bash}]{stdout} to 
      print 
      a) any intermediate output you deem relevant, followed finally by 
      b) two lines which clearly detail the target material recovered plus
         the total number of interactions with the attack target.
\item Answer the exam-style questions in  
      \lstinline[language={bash}]|${ARCHIVE}/CONF(ARCHIVE_PATH,CID)/${USER}.exam|.
\end{enumerate}

% =============================================================================
