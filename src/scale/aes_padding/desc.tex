% Copyright (C) 2017 Daniel Page <dan@phoo.org>
%
% Use of this source code is restricted per the CC BY-SA license, a copy of
% which can be found via http://creativecommons.org (and should be included 
% as LICENSE.txt within the associated archive or repository).

% =============================================================================

\leveldown{Background}

Imagine you encounter a server, denoted 
$\PARTY{T}$, 
which acts as a storage repository for sensor data.  An installation of $50$ 
or so IoT-class sensor nodes regularly transmit encrypted packets of data to
$\PARTY{T}$, 
which stores them for later analysis (iff. they are deemed to be valid).  
Having already captured some encrypted packets transmitted by a target sensor
node, you are tasked with recovery of the underlying sensor data.
\DESCINTRO[none]
{% Copyright (C) 2017 Daniel Page <dan@phoo.org>
%
% Use of this source code is restricted per the CC BY-SA license, a copy of
% which can be found via http://creativecommons.org (and should be included 
% as LICENSE.txt within the associated archive or repository).

\begin{tikzpicture}

% =============================================================================

% configure
 \tikzset{node distance={6cm},party/.style={rectangle,minimum width={2.0cm},minimum height={2.0cm}}}

% adversary and target device
 \node [                party] (E) {$\PARTY{E}$} ;
 \node [right of=E,draw,party] (T) {$\PARTY{T}$} ;

% embedded material
 \node at (T.south east) [anchor=south east,draw,rectangle] {\tiny $\PRI{\CHAL{k}}$} ;

%   direct input and output
 \draw [       >=stealth,->] (E.30)    to              
                             node [above]             {$\PUB{m}$} 
                             (E.30    -| T.west)  ;
 \draw [       >=stealth,<-] (E.330)   to              
                             node [below]             {$\PUB{c}$} 
                             (E.330   -| T.west)  ;

% indirect input and output
 \draw [dashed,>=stealth,<-] (T.north) to [bend right] 
                             node [above]             {$\FAULT \models \mbox{clock glitch}$}
                             (T.north -| E.north) ;
%\draw [dashed,>=stealth,->] (T.south) to [bend left]  
%                            node [below]             {} 
%                            (T.south -| E.south) ;

% computation
 \draw [       >=stealth,->] (T.30)    to [bend left,in=90,out=90,looseness=2] 
                             node [right,anchor=west] {$\PUB{c} = \SCOPE{\ID{AES-128}}{\ALG{Enc}}( \PRI{\CHAL{k}}, \PUB{m} )$}
                             (T.330) ;

% =============================================================================

\end{tikzpicture}

}
{some key material}
{a ciphertext}
{i.e., it processes the ciphertext, as detailed further below}
{}
In more detail, the algorithm
$\ALG{Process}$
which captures the computation performed by
$\PARTY{T}$
can be described by the following steps:

\begin{enumerate}
\item decrypt the $\PUB{l}$-block $\PUB{c}$, i.e., compute
      \[
      \PRI{m} \CONS \PRI{\TAG} \CONS \PRI{\PAD} = \SCOPE{\ID{AES-128-CBC}}{\ALG{Dec}}( \PRI{\CHAL{k}_1}, \PUB{iv}, \PUB{c} )
      \]
      then
\item check whether $\PRI{\PAD}$ is valid, 
      aborting immediately  and produce result code $1$ if not,
      then
\item check whether $\PRI{\TAG}$ is valid, i.e., whether 
      \[
      \SCOPE{\ID{HMAC-SHA-1}}{\ALG{Ver}}( \PRI{\CHAL{k}_2}, \PRI{m}, \PRI{\TAG} ) = \TRUE ,
      \]
      aborting immediately  and produce result code $2$ if not,
      then
\item process $\PRI{m}$ somehow,
                            and produce result code $0$.
\end{enumerate}

\noindent
Since there is no (external) output from
$\PARTY{T}$
(e.g., relating to $\PRI{r}$), 
one might question how
$\PARTY{E}$
obtains the (internal) result code.  Doing so is plausible, because 
$\PARTY{E}$
can use execution latency as a proxy: since computation by
$\PARTY{T}$
early-aborts depending on the error condition, observation of the execution
latency is (modulo any experimental noise) suggestive of whether, when, and 
what error has occured.

% -----------------------------------------------------------------------------

\levelstay{Material}

\leveldown{\lstinline[language={bash}]|\$\{ARCHIVE\}/CONF(ARCHIVE_PATH,CID)/\$\{USER\}.T|}

\DESCMATERIALT{%
\item $\PUB{l}$,
      a  length
      (represented as a                       decimal integer string),
\item $\PUB{c}$,
      an $\PUB{l}$-block AES-128-CBC ciphertext
      (represented as a  length-prefixed, hexadecimal octet   string),
      and
\item $\PUB{iv}$,
      a      ${1}$-block AES-128-CBC initialisation vector
      (represented as a  length-prefixed, hexadecimal octet   string),
}{%
\item $\LEAK$,
      a  result code
      (represented as a                       decimal integer string),
}%
Note that:

\begin{itemize}
\item To avoid attacks that leverage execution latency, 
      $\PARTY{T}$ 
      employs a high-performance yet constant-time AES-128 implementation 
      (operated in CBC mode) that is derived from~\cite{SCALE:KasSch:09};
      this implies $128$-bit block and cipher key lengths.  
\item Keep in mind some limitations, namely 
      $
      0 \leq \PUB{l} < 256 ,
      $
      on maximum plaintext and/or ciphertext length.
\item The \PKCS{7}[1.5]~\cite[Section 10.3]{SCALE:RFC:2315} padding scheme 
      is used: given a block size of $16$ bytes, define
      $
      p = 16 - ( | \PRI{m} \CONS \PRI{\TAG} | \bmod 16 ) 
      $
      which implies that
      $
      1 \leq p \leq 16 .
      $
      The padding can then be described as
      \[
      \PRI{\PAD} = \LIST{ \underbrace{ p, p, \ldots, p }_{\mbox{$p$ octets}} } ,
      \]
      i.e., a sequence of $p$ octets each of whose value is $p$.
\end{itemize}

\levelstay{\lstinline[language={bash}]|\$\{ARCHIVE\}/CONF(ARCHIVE_PATH,CID)/\$\{USER\}.conf|}

\DESCMATERIALCONF{%
\item $\PUB{\CHAL{c}}$,
      a  ${1}$-block AES-128-CBC ciphertext 
      (represented as a  length-prefixed, hexadecimal octet   string),
      corresponding to an encryption of some unknown plaintext 
      $\PRI{\CHAL{m}}$ (using $\PUB{\CHAL{iv}}$),
      and
\item $\PUB{\CHAL{iv}}$,
      a  ${1}$-block AES-128-CBC initialisation vector
      (represented as a  length-prefixed, hexadecimal octet   string),
}%
More specifically, this represents the previously captured network traffic
whose decryption,
i.e., recovery of the underlying plaintext $\PRI{\CHAL{m}}$,
is the task at hand.
\IfStrEqCase{CONF(CHALLENGE,CID)}{%
  {0}{Keep in mind that $\PRI{\CHAL{m}}$ will 
      include the SHA-1 hash          of \lstinline[language={bash}]{$\{USER\}}
      as the least-significant octets: 
      this allows candidate decryptions to be checked for validity.
  }%
  {1}{Keep in mind that $\PRI{\CHAL{m}}$ will 
      include an ASCII representation of \lstinline[language={bash}]{$\{USER\}} 
      as the least-significant octets: 
      this allows candidate decryptions to be checked for validity.
  }%
  {2}{Keep in mind that $\PRI{\CHAL{m}}$ will have been
      be generated entirely at random:
      this means checking validity of candidate decryptions is more 
      difficult than it would be otherwise.
  }%
}%

% -----------------------------------------------------------------------------

\levelup{Tasks}

\begin{enumerate}
\item \DESCTASKIMPL
      {$\PRI{\CHAL{m}}$}
      {\mbox{\lstinline[language={bash}]|./attack \$\{USER\}.T \$\{USER\}.conf|}}
\item \DESCTASKEXAM
      {\mbox{\lstinline[language={bash}]|\$\{ARCHIVE\}/CONF(ARCHIVE_PATH,CID)/\$\{USER\}.exam|}}
\end{enumerate}

% =============================================================================
