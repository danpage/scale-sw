% Copyright (C) 2017 Daniel Page <dan@phoo.org>
%
% Use of this source code is restricted per the CC BY-SA license, a copy of
% which can be found via http://creativecommons.org (and should be included 
% as LICENSE.txt within the associated archive or repository).

% =============================================================================

\leveldown{Background}

\PKCS{1}[2.1]~\cite{SCALE:RFC:3447}
specifies the
RSAES-OAEP decryption~\cite[Section 7.1]{SCALE:RFC:3447}
scheme, which, in turn, makes use of 
Optimal Asymmetric Encryption Padding (OAEP)~\cite{SCALE:BelRog:94}.  
Crucially, the decryption process can encounter various error conditions:

\begin{description}
\item[Error \NUM{1}:]
     A decryption error occurs in Step $3.g$ of \ALG{RSAES-OAEP-Decrypt} 
     if the octet string passed to the decoding phase
     does {\em not} have a most-significant $\RADIX{00}{16}$ octet.
     Put another way, the error occurs because 
     the output produced by RSA decryption is too large to fit into one 
     fewer octets than the modulus.
\item[Error \NUM{2}:]
     A decryption error occurs in Step $3.g$ of \ALG{RSAES-OAEP-Decrypt} 
     if the octet string passed into the decoding phase 
     does {\em not} a) produce a hashed label that matches, or b) use a
     $\RADIX{01}{16}$ octet between any padding and the message.
     Put another way, the error occurs because 
     the plaintext validity checking mechanism fails.
\end{description}

\noindent
A footnote~\cite[Section 7.1.2]{SCALE:RFC:3447} explains why all errors, 
these two in particular, should be indistinguishable from each other; a
given application should not reveal {\em which} error occurred, even if
it reveals {\em some} error occurred.  

Imagine you form part of a red team\footnote{
\url{https://en.wikipedia.org/wiki/Red_team}
} asked to assess a specific e-commerce server, denoted 
$\PARTY{T}$, 
which houses a $64$-bit Intel Core2 micro-processor.  
In reality, the server is a HSM-like device representing the back-office 
infrastructure that supports various secure web-sites: the server offers
secure key generation and storage, plus off-load for some cryptographic 
operations.  
$\PARTY{T}$ 
is able to compute RSAES-OAEP decryption, but, crucially, does {\em not} 
adhere to the advice re. error indistinguishability outlined above.
\DESCINTRO[leak]
{a private key}
{a ciphertext}
{i.e., it processes the ciphertext, as detailed further below}
{} % no output!
In more detail, the algorithm
$\ALG{Process}$
which captures the computation performed by
$\PARTY{T}$
can be described by the following steps.

\begin{enumerate}
\item decrypt $\PUB{c}$, i.e., compute
      \[
      \PRI{m} = \SCOPE{\ID{RSAES-OAEP}}{\ALG{Decrypt}}( \TUPLE{ \PUB{\CHAL{N}}, \PRI{\CHAL{d}} }, \PUB{c} ) ,
      \]
      and, in doing so, deal with various error conditions:

      \begin{itemize}
      \item if error \NUM{1} occurred during decryption,
            abort immediately  and produce   result code $1$,
      \item if error \NUM{2} occurred during decryption,
            abort immediately  and produce   result code $2$,
      \item if there was some other internal error 
            (e.g., due to malformed input),
            abort immediately  and produce a result code that attempts to diagnose the cause:

            \begin{itemize}
            \item if the result code is $3$ then \ALG{RSAEP}              
                  failed because the operand was out of range 
                  (section $5.1.1$, step $1$, page $11$), 
                  i.e., the  plaintext is not between $0$ and $\PUB{\CHAL{N}}-1$,
            \item if the result code is $4$ then \ALG{RSADP}              
                  failed because the operand was out of range 
                  (section $5.1.2$, step $1$, page $11$), 
                  i.e., the ciphertext is not between $0$ and $\PUB{\CHAL{N}}-1$,
            \item if the result code is $5$ then \ALG{RSAES-OAEP-Encrypt} 
                  failed because a length check failed        
                  (section $7.1.1$, step $1.b$, page $18$), 
                  i.e., the message is too long,
            \item if the result code is $6$ then \ALG{RSAES-OAEP-Decrypt} 
                  failed because a length check failed        
                  (section $7.1.2$, step $1.b$, page $20$), 
                  i.e., the ciphertext does not match the length of $\PUB{\CHAL{N}}$,
            \item if the result code is $7$ then \ALG{RSAES-OAEP-Decrypt} 
                  failed because a length check failed        
                  (section $7.1.2$, step $1.c$, page $20$), 
                  i.e., the ciphertext does not match the length of the hash function output.
            \item any other result code 
                  (of $8$ upward) 
                  implies an abnormal error whose cause cannot be directly 
                  associated with any of the above.
            \end{itemize}
      
            \noindent
            
      \end{itemize}

\item process $\PRI{m}$ somehow,
                               and produce   result code $0$.          
\end{enumerate}

\noindent
Since there is no (external) output, one might question how
$\PARTY{E}$
obtains the (internal) result code.  Doing so is plausible, because 
$\PARTY{E}$
can use execution latency as a proxy: since computation by
$\PARTY{T}$
early-aborts depending on the error condition, observation of the execution
latency is (modulo any experimental noise) suggestive of whether, when, and 
what error has occured.

% -----------------------------------------------------------------------------

\levelstay{Material}

\leveldown{\lstinline[language={bash}]|\$\{ARCHIVE\}/CONF(ARCHIVE_PATH,CID)/\$\{USER\}.T|}

\DESCMATERIALT{%
\item $\PUB{l}$,
      an RSAES-OAEP label
      (represented as a  length-prefixed, hexadecimal octet   string),
      and
\item $\PUB{c}$,
      an RSAES-OAEP ciphertext
      (represented as a  length-prefixed, hexadecimal octet   string),
}{%
\item $\LEAK$,
      a  result code
      (represented as a                       decimal integer string),
}%

\levelstay{\lstinline[language={bash}]|\$\{ARCHIVE\}/CONF(ARCHIVE_PATH,CID)/\$\{USER\}.conf|}

\DESCMATERIALCONF{%
\item $\PUB{\CHAL{N}}$,     
      an RSA        modulus
      (represented as a                   hexadecimal integer string),
\item $\PUB{\CHAL{e}}$,
      an RSA        public exponent
      (represented as a                   hexadecimal integer string),
      such that $\PUB{\CHAL{e}} \cdot \PRI{\CHAL{d}} \equiv 1 \pmod{\PRI{\Phi{(\CHAL{N})}}}$,
\item $\PUB{\CHAL{l}}$,
      an RSAES-OAEP label
      (represented as a  length-prefixed, hexadecimal octet   string),
      and
\item $\PUB{\CHAL{c}}$,
      an RSAES-OAEP ciphertext 
      (represented as a  length-prefixed, hexadecimal octet   string),
      corresponding to an encryption of some unknown plaintext 
      $\PRI{\CHAL{m}}$ (using $\PUB{\CHAL{l}}$),
}%
Note that
$
\TUPLE{ \PUB{\CHAL{N}}, \PUB{\CHAL{e}} }
$
is
a     (known) RSA public  key 
associated with 
$
\TUPLE{ \PUB{\CHAL{N}}, \PRI{\CHAL{d}} } ,
$
i.e.,
the (unknown) RSA private key,
and that
$
\PUB{\CHAL{c}}
$
is a ciphertext 
whose decryption,
i.e., recovery of the underlying plaintext \PRI{\CHAL{m}},
is the task at hand.
You can assume the RSAES-OAEP encryption of 
$
\PRI{\CHAL{m}}
$ 
that produced 
$
\PUB{\CHAL{c}}
$
used the \ALG{MGF1} mask generation function with 
\IfStrEqCase{CONF(HASH,CID)}{%
  {0}  {SHA-1 }% SHA-1
  {1}{SHA-256 }% SHA-256
  {2}{SHA-384 }% SHA-384
  {3}{SHA-512 }% SHA-512
} as the underlying hash function.
\IfStrEqCase{CONF(CHALLENGE,CID)}{%
  {0}{Additionally, 
      $\PRI{\CHAL{m}}$ will 
      include the SHA-1 hash          of \lstinline[language={bash}]{$\{USER\}}
      as the least-significant octets: 
      this allows candidate decryptions to be checked for validity.
  }%
  {1}{Additionally, 
      $\PRI{\CHAL{m}}$ will 
      include an ASCII representation of \lstinline[language={bash}]{$\{USER\}} 
      as the least-significant octets: 
      this allows candidate decryptions to be checked for validity.
  }%
  {2}{Additionally, 
      $\PRI{\CHAL{m}}$ will have been
      be generated entirely at random:
      this means checking validity of candidate decryptions is more 
      difficult than it would be otherwise.
  }%
}%

% -----------------------------------------------------------------------------

\levelup{Tasks}

\begin{enumerate}
\item \DESCTASKIMPL
      {$\PRI{\CHAL{m}}$}
      {\mbox{\lstinline[language={bash}]|./attack \$\{USER\}.T \$\{USER\}.conf|}}
\item \DESCTASKEXAM
      {\mbox{\lstinline[language={bash}]|\$\{ARCHIVE\}/CONF(ARCHIVE_PATH,CID)/\$\{USER\}.exam|}}
\end{enumerate}

% =============================================================================
