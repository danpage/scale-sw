% Copyright (C) 2017 Daniel Page <dan@phoo.org>
%
% Use of this source code is restricted per the CC BY-SA license, a copy of
% which can be found via http://creativecommons.org (and should be included 
% as LICENSE.txt within the associated archive or repository).

% =============================================================================

\leveldown{Background}

Imagine you encounter a server, denoted 
$\PARTY{T}$, 
which houses an ageing but still operational $32$-bit Katmai model Intel 
Pentium III micro-processor.  
A datasheet for this micro-processor shows that it has a $16$ kilobyte, 
$4$-way set-associative L1 data cache with $32$ byte cache lines.  Since
around $1999$, the server has performed a mission-critical back-office 
role in a banking application: you discover it acts as a primitive form 
of HSM, performing DES encryption on behalf of front-end clients.
\DESCINTRO[leak]
{% Copyright (C) 2017 Daniel Page <dan@phoo.org>
%
% Use of this source code is restricted per the CC BY-SA license, a copy of
% which can be found via http://creativecommons.org (and should be included 
% as LICENSE.txt within the associated archive or repository).

\begin{tikzpicture}

% =============================================================================

% configure
 \tikzset{node distance={6cm},party/.style={rectangle,minimum width={2.0cm},minimum height={2.0cm}}}

% adversary and target device
 \node [                party] (E) {$\PARTY{E}$} ;
 \node [right of=E,draw,party] (T) {$\PARTY{T}$} ;

% embedded material
 \node at (T.south east) [anchor=south east,draw,rectangle] {\tiny $\PRI{\CHAL{k}}$} ;

%   direct input and output
 \draw [       >=stealth,->] (E.30)    to              
                             node [above]             {$\PUB{m}$} 
                             (E.30    -| T.west)  ;
 \draw [       >=stealth,<-] (E.330)   to              
                             node [below]             {$\PUB{c}$} 
                             (E.330   -| T.west)  ;

% indirect input and output
 \draw [dashed,>=stealth,<-] (T.north) to [bend right] 
                             node [above]             {$\FAULT \models \mbox{clock glitch}$}
                             (T.north -| E.north) ;
%\draw [dashed,>=stealth,->] (T.south) to [bend left]  
%                            node [below]             {} 
%                            (T.south -| E.south) ;

% computation
 \draw [       >=stealth,->] (T.30)    to [bend left,in=90,out=90,looseness=2] 
                             node [right,anchor=west] {$\PUB{c} = \SCOPE{\ID{AES-128}}{\ALG{Enc}}( \PRI{\CHAL{k}}, \PUB{m} )$}
                             (T.330) ;

% =============================================================================

\end{tikzpicture}

}
{a cipher key}
{a  plaintext}
{i.e., it encrypts the plaintext using the cipher key}
{a ciphertext}
Note that the ability to observe execution latency is plausible, because
$\PARTY{E}$ 
can simply record how long
$\PARTY{T}$ 
takes to respond with the output associated with a given input (noting the potential for experimental noise while doing so).

% -----------------------------------------------------------------------------

\levelstay{Material}

\leveldown{\lstinline[language={bash}]|\$\{ARCHIVE\}/CONF(ARCHIVE_PATH,CID)/\$\{USER\}.T|}

\DESCMATERIALT{%
\item $\PUB{m}$, 
      a  ${1}$-block DES plaintext 
      (represented as a  length-prefixed, hexadecimal octet   string), 
}{%
\item $\LEAK$,
      an execution latency measured in clock cycles
      (represented as a                       decimal integer string),
      and
\item $\PUB{c}$,
      a  ${1}$-block DES ciphertext
      (represented as a  length-prefixed, hexadecimal octet   string), 
}%
Note that:

\begin{itemize}
\item $\PARTY{T}$ uses a software implementation of DES, based on that
      written by Richard Outerbridge~\cite[Part V]{SCALE:Schneier:06}.
\item DES has a $64$-bit block length, and, although it notionally accepts 
      a $64$-bit cipher key, the {\em effective} cipher key length is in
      fact $56$ bits: the remaining $8$ bits support error detection via 
      a parity code.  You can ignore the parity bits, meaning {\em many}
      $64$-bit candidate cipher keys will be valid when used in place of 
      $\PRI{\CHAL{k}}$ (since only the $56$ bits actually used by DES are 
      relevant).
\item The cited source code uses a set of eight separate S-box look-up 
      tables.  Each look-up table has $64$ entries of type 
      \lstinline[language=C]{unsigned long} 
      (i.e., a $64$-bit unsigned integer), so will consume $512$ bytes 
      of memory.  Doing so improves efficiency: a look-up table captures 
      the function of an associated S-box {\em plus} the (inefficient) 
      P- and E-permutations.  However, the implementation actually used 
      by $\PARTY{T}$ differs slightly.  Given each of the look-up table 
      entries holds $32$ bits of content, the type of entries has been 
      altered to 
      \lstinline[language=C]{unsigned int}. 
      Once loaded, an entry is simply cast into 
      \lstinline[language=C]{unsigned long},
      roughly halving the memory footprint with no impact on efficiency.
      Given the $32$-byte L1 cache line size, and that each look-up table 
      is aligned to a $32$-byte boundary, each cache line can therefore 
      accommodate eight full $32$-bit entries.
\item To perform the role described, the implementation

      \begin{enumerate}
      \item invokes the \lstinline[language=C]{deskey} function
            to pre-compute the round keys from a given cipher key,
            then
      \item invokes the \lstinline[language=C]{des}    function
            to perform an encryption operation.
      \end{enumerate}

      \noindent
      Within the 
      \lstinline[language=C]{des}
      function, 
      \lstinline[language=C]{scrunch} and \lstinline[language=C]{unscrun}
      are used to convert a sequence of eight $8$-bit bytes to and from a
      $64$-bit integer (held as two $32$-bit halves) before and after 
      \lstinline[language=C]{desfunc}
      performs the encryption operation itself.
\item You can assume that when an encryption operation is performed, i.e.,
      when the
      \lstinline[language=C]{des}
      function is invoked, the cache contains {\em no} S-box content: any
      cache-hits or -misses during encryption therefore depend only on 
      $\PRI{\CHAL{k}}$ and $\PUB{m}$, as does the execution latency.
\end{itemize}

\levelstay{\lstinline[language={bash}]|\$\{ARCHIVE\}/CONF(ARCHIVE_PATH,CID)/\$\{USER\}.R|}

\DESCMATERIALR{%
\item $\PUB{\REPL{m}}$, 
      a  $1$-block DES plaintext
      (represented as a  length-prefixed, hexadecimal octet   string), 
      and
\item $\PUB{\REPL{k}}$, 
      a           DES cipher key
      (represented as a  length-prefixed, hexadecimal octet   string), 
}%
In contrast to 
$\PARTY{T}$, 
this means 
$\PARTY{R}$ 
will use the
{\em chosen} DES cipher key $\PUB{\REPL{k}}$
to encrypt the
     chosen  DES  plaintext $\PUB{\REPL{m}}$.

% -----------------------------------------------------------------------------

\levelup{Tasks}

\begin{enumerate}
\item \DESCTASKIMPL
      {$\PRI{\CHAL{k}}$}
      {\mbox{\lstinline[language={bash}]|./attack \$\{USER\}.T|}}
\item \DESCTASKEXAM
      {\mbox{\lstinline[language={bash}]|\$\{ARCHIVE\}/CONF(ARCHIVE_PATH,CID)/\$\{USER\}.exam|}}
\end{enumerate}

% =============================================================================
