% Copyright (C) 2017 Daniel Page <dan@phoo.org>
%
% Use of this source code is restricted per the CC BY-SA license, a copy of
% which can be found via http://creativecommons.org (and should be included 
% as LICENSE.txt within the associated archive or repository).

% =============================================================================

\leveldown{Background}

Imagine you are tasked with attacking a device, denoted 
$\PARTY{T}$, 
which houses an $8$-bit Intel $8051$ micro-processor.  More specifically,
$\PARTY{T}$ 
represents an ISO/IEC 7816 compliant contact-based smart-card used within 
larger devices as a cryptographic co-processor module: 
using a standardised protocol (or API), it supports secure key generation 
and storage plus off-load of some cryptographic operations.
\DESCINTRO[leak]
{% Copyright (C) 2017 Daniel Page <dan@phoo.org>
%
% Use of this source code is restricted per the CC BY-SA license, a copy of
% which can be found via http://creativecommons.org (and should be included 
% as LICENSE.txt within the associated archive or repository).

\begin{tikzpicture}

% =============================================================================

% configure
 \tikzset{node distance={6cm},party/.style={rectangle,minimum width={2.0cm},minimum height={2.0cm}}}

% adversary and target device
 \node [                party] (E) {$\PARTY{E}$} ;
 \node [right of=E,draw,party] (T) {$\PARTY{T}$} ;

% embedded material
 \node at (T.south east) [anchor=south east,draw,rectangle] {\tiny $\PRI{\CHAL{k}}$} ;

%   direct input and output
 \draw [       >=stealth,->] (E.30)    to              
                             node [above]             {$\PUB{m}$} 
                             (E.30    -| T.west)  ;
 \draw [       >=stealth,<-] (E.330)   to              
                             node [below]             {$\PUB{c}$} 
                             (E.330   -| T.west)  ;

% indirect input and output
 \draw [dashed,>=stealth,<-] (T.north) to [bend right] 
                             node [above]             {$\FAULT \models \mbox{clock glitch}$}
                             (T.north -| E.north) ;
%\draw [dashed,>=stealth,->] (T.south) to [bend left]  
%                            node [below]             {} 
%                            (T.south -| E.south) ;

% computation
 \draw [       >=stealth,->] (T.30)    to [bend left,in=90,out=90,looseness=2] 
                             node [right,anchor=west] {$\PUB{c} = \SCOPE{\ID{AES-128}}{\ALG{Enc}}( \PRI{\CHAL{k}}, \PUB{m} )$}
                             (T.330) ;

% =============================================================================

\end{tikzpicture}

}
{a cipher key}
{a  plaintext}
{i.e., it encrypts the plaintext using the cipher key}
{a ciphertext}
Note that the ability to observe 
execution latency 
is plausible, because
$\PARTY{E}$ 
can simply record how long
$\PARTY{T}$ 
takes to respond with the output associated with a given input (noting the potential for experimental noise while doing so).

% -----------------------------------------------------------------------------

\levelstay{Material}

\leveldown{\lstinline[language={bash}]|\$\{ARCHIVE\}/CONF(ARCHIVE_PATH,CID)/\$\{USER\}.T|}

\DESCMATERIALT{%
\item $\PUB{m}$,
      a  ${1}$-block AES  plaintext
      (represented as a  length-prefixed, hexadecimal octet   string),
}{%
\item $\LEAK$,
      an execution latency measured in clock cycles
      (represented as a                       decimal integer string),
      and
\item $\PUB{c}$,
      a  ${1}$-block AES ciphertext
      (represented as a  length-prefixed, hexadecimal octet   string),
}%
Note that:

\begin{itemize}
\item $\PARTY{T}$ 
      uses an AES-128 implementation, which implies $128$-bit block and 
      cipher key lengths;
      following the notation in~\cite[Figure 5]{SCALE:FIPS:197:01}, note
      that 
      $
      Nb =  4
      $ 
      and 
      $
      Nr = 10
      $ 
      means a $( 4 \times 4 )$-element state matrix is used in a total of 
      $11$ rounds.
\item More concretely, it is an $8$-bit, memory-constrained implementation
      with the S-box held as a $\SI{256}{\byte}$ look-up table in memory. 
      In line with the goal of minimising the memory footprint, the round
      keys are not pre-computed: each encryption takes the cipher key and
      evolves it forward, step-by-step, to form successive round keys for
      use during key addition.  

      Crucially, the implementation realises 
      \lstinline[language={C}]|xtime| (i.e., multiplication-by-$\IND{x}$,
      as used in the \lstinline{MixColumns} round function) by computing
      it via

      \begin{lstlisting}[language={C},gobble={6},frame={single},basicstyle={\ttfamily\small}]
      uint8_t xtime( uint8_t a ) {
        if( a & 0x80 ) {
          return 0x1B ^ ( a << 1 );
        }
        else {
          return        ( a << 1 );
        }
      }
      \end{lstlisting}

      \noindent
      rather than, e.g., pre-computed and then realised via accesses into 
      a look-up table.  Note that the resulting computation is {\em not} 
      data-independent.
\end{itemize}

\levelstay{\lstinline[language={bash}]|\$\{ARCHIVE\}/CONF(ARCHIVE_PATH,CID)/\$\{USER\}.R|}

\DESCMATERIALR{%
\item $\PUB{\REPL{m}}$,
      a  ${1}$-block     AES  plaintext
      (represented as a  length-prefixed, hexadecimal octet   string),
      and            
\item $\PUB{\REPL{k}}$,
      an                 AES cipher key
      (represented as a  length-prefixed, hexadecimal octet   string),
}%
In contrast to 
$\PARTY{T}$, 
this means 
$\PARTY{R}$ 
will use the
{\em chosen} cipher key $\PUB{\REPL{k}}$ 
to encrypt the
     chosen  plaintext  $\PUB{\REPL{m}}$.

% -----------------------------------------------------------------------------

\levelup{Tasks}

\begin{enumerate}
\item \DESCTASKIMPL
      {$\PRI{\CHAL{k}}$}
      {\mbox{\lstinline[language={bash}]|./attack \$\{USER\}.T|}}
\item \DESCTASKEXAM
      {\mbox{\lstinline[language={bash}]|\$\{ARCHIVE\}/CONF(ARCHIVE_PATH,CID)/\$\{USER\}.exam|}}
\end{enumerate}

% =============================================================================
