% Copyright (C) 2017 Daniel Page <dan@phoo.org>
%
% Use of this source code is restricted per the CC BY-SA license, a copy of
% which can be found via http://creativecommons.org (and should be included 
% as LICENSE.txt within the associated archive or repository).

% =============================================================================

\leveldown{Background}

A given device, denoted $\PARTY{T}$, 
implements a basic access control mechanism: in order to access the device
(e.g., log into it), a user must first provide the correct password (i.e., 
a guess matching the actual password embedded in $\PARTY{T}$).  
Once in posession of the device, any such user, e.g., 
an attacker $\PARTY{E}$, can interact with it as follows:

\begin{center}
% Copyright (C) 2017 Daniel Page <dan@phoo.org>
%
% Use of this source code is restricted per the CC BY-SA license, a copy of
% which can be found via http://creativecommons.org (and should be included 
% as LICENSE.txt within the associated archive or repository).

\begin{tikzpicture}

% =============================================================================

% configure
 \tikzset{node distance={6cm},party/.style={rectangle,minimum width={2.0cm},minimum height={2.0cm}}}

% adversary and target device
 \node [                party] (E) {$\PARTY{E}$} ;
 \node [right of=E,draw,party] (T) {$\PARTY{T}$} ;

% embedded material
 \node at (T.south east) [anchor=south east,draw,rectangle] {\tiny $\PRI{\CHAL{k}}$} ;

%   direct input and output
 \draw [       >=stealth,->] (E.30)    to              
                             node [above]             {$\PUB{m}$} 
                             (E.30    -| T.west)  ;
 \draw [       >=stealth,<-] (E.330)   to              
                             node [below]             {$\PUB{c}$} 
                             (E.330   -| T.west)  ;

% indirect input and output
 \draw [dashed,>=stealth,<-] (T.north) to [bend right] 
                             node [above]             {$\FAULT \models \mbox{clock glitch}$}
                             (T.north -| E.north) ;
%\draw [dashed,>=stealth,->] (T.south) to [bend left]  
%                            node [below]             {} 
%                            (T.south -| E.south) ;

% computation
 \draw [       >=stealth,->] (T.30)    to [bend left,in=90,out=90,looseness=2] 
                             node [right,anchor=west] {$\PUB{c} = \SCOPE{\ID{AES-128}}{\ALG{Enc}}( \PRI{\CHAL{k}}, \PUB{m} )$}
                             (T.330) ;

% =============================================================================

\end{tikzpicture}


\end{center}

\noindent
That is, in each interaction $\PARTY{E}$ can (adaptively) send 
a chosen password guess
to $\PARTY{T}$; the device will
perform a comparison with the fixed, unknown password $\PRI{\CHAL{P}}$ 
and produce 
a result code detailing whether or not the two match.
In addition, $\PARTY{E}$ can measure the time $\PARTY{T}$ takes to execute 
each operation: it does so
by simply timing how long $\PARTY{T}$ takes to respond with $\PUB{r}$ once
provided with $\PUB{G}$.

% -----------------------------------------------------------------------------

\levelstay{Material}

\leveldown{\lstinline[language={bash}]|\$\{ARCHIVE\}/CONF(ARCHIVE_PATH,CID)/\$\{USER\}.T|}

This executable simulates the attack target $\PARTY{T}$.  When executed it 
 reads the following  input (one field per line)

\begin{itemize}
\item $\PUB{G}$,
      a  password guess
      \IfStrEqCase{CONF(ALPHABET,CID)}{%
        {0}{%
          (represented as a  length-prefixed, hexadecimal octet   string),
        }%
        {1}{%
          (represented as an                              ASCII   string),
        }%
      }%
\end{itemize}

\noindent
from \lstinline[language={bash}]{stdin},
and 
writes the following output (one field per line)

\begin{itemize}
\item $\LEAK$,
      an execution time measured in clock cycles
          (represented as a                       decimal integer string),
      and
\item $\PUB{r}$,
      a  result code
          (represented as a                       decimal integer string),
\end{itemize}

\noindent
to   \lstinline[language={bash}]{stdout}.
Execution continues this way, i.e., repeatedly reading input then writing 
output, until $\PARTY{T}$ is forcibly terminated (or crashes).
Note that:

\begin{itemize}
\item Given the scenario, notice that even if $\PUB{r}$ were not produced 
      as output, $\PARTY{E}$ could still learn whether or not their guess 
      was correct: if the guess is incorrect then access to $\PARTY{T}$ 
      will denied, whereas if the guess is correct then access is granted.
\item The comparison between $\PRI{\CHAL{P}}$ and $\PUB{G}$ is implemented 
      by $\PARTY{T}$ as follows:

      \begin{algorithm}[H]
      \If{$|\PRI{\CHAL{P}}| \neq |\PUB{G}|$} {
        \KwRet{$0$}\;
      }
      \For{$i$ {\bf from} $0$ {\bf upto} $|\PRI{\CHAL{P}}|$} {
        \If{$\PRI{\CHAL{P}_i} \neq \PUB{G_i}$} {
          \KwRet{$0$}\; 
        }
      }
      \KwRet{$1$}\; 
      \end{algorithm}
\end{itemize}

\levelstay{\lstinline[language={bash}]|\$\{ARCHIVE\}/CONF(ARCHIVE_PATH,CID)/\$\{USER\}.R|}

In many side-channel and fault attacks, the attacker is able to perform an 
initial profiling or calibration phase: a common rationale for doing so is 
to support selection or fine-tuning of parameters for a subsequent attack.  
With this in mind, a simulated replica $\PARTY{R}$ of the attack target is 
also provided.
$\PARTY{R}$ is identical to $\PARTY{T}$, bar the fact it
 reads the following  input (one field per line)

\begin{itemize}
\item $\PUB{\REPL{G}}$,
      a  password guess
      \IfStrEqCase{CONF(ALPHABET,CID)}{%
        {0}{%
          (represented as a  length-prefixed, hexadecimal octet   string),
        }%
        {1}{%
          (represented as an                              ASCII   string),
        }%
      }%
      and
\item $\PUB{\REPL{P}}$,
      a  password
      \IfStrEqCase{CONF(ALPHABET,CID)}{%
        {0}{%
          (represented as a  length-prefixed, hexadecimal octet   string),
        }%
        {1}{%
          (represented as an                              ASCII   string),
        }%
      }%
\end{itemize}

\noindent
on \lstinline[language={bash}]{stdin}:
in contrast to $\PARTY{T}$, this means $\PARTY{R}$ uses the
{\em chosen} password $\PUB{\REPL{P}}$
for comparison with the
     chosen  guess    $\PUB{\REPL{G}}$.

% -----------------------------------------------------------------------------

\levelup{Tasks}

      Write a program that simulates the adversary $\PARTY{E}$ by attacking
      the simulated target, or, more specifically, that recovers the target 
      material $\PRI{\CHAL{P}}$.  
      When executed using a command of the form
      \[
      \mbox{\lstinline[language={bash}]|./attack \$\{USER\}.T|}
      \]
      the attack should be invoked on the simulated target named (not some
      hard-coded alternative).  Use \lstinline[language={bash}]{stdout} to 
      print 
      a) any intermediate output you deem relevant, followed finally by 
      b) two lines which clearly detail the target material recovered plus
         the total number of interactions with the attack target.

% =============================================================================
