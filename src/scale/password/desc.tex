% Copyright (C) 2017 Daniel Page <dan@phoo.org>
%
% Use of this source code is restricted per the CC BY-SA license, a copy of
% which can be found via http://creativecommons.org (and should be included 
% as LICENSE.txt within the associated archive or repository).

% =============================================================================

\leveldown{Background}

A given device, denoted 
$\PARTY{T}$, 
implements a basic access control mechanism: in order to access the device
(e.g., log into it), 
a user must first provide the correct password.
\DESCINTRO[leak]
{a password}
{a password attempt}
{i.e., it validates the password attempt by comparing it with the correct, stored password}
{a result code}
Note that the ability to observe execution latency is plausible, because
$\PARTY{E}$ 
can simply record how long
$\PARTY{T}$ 
takes to respond with the output associated with a given input (noting the potential for experimental noise while doing so).

% -----------------------------------------------------------------------------

\levelstay{Material}

\leveldown{\lstinline[language={bash}]|\$\{ARCHIVE\}/CONF(ARCHIVE_PATH,CID)/\$\{USER\}.T|}

\DESCMATERIALT{%
\item $\PUB{G}$,
      a  password attempt
      \IfStrEqCase{CONF(ALPHABET,CID)}{%
        {0}{%
          (represented as a  length-prefixed, hexadecimal octet   string),
        }%
        {1}{%
          (represented as an                              ASCII   string),
        }%
      }%
}{%
\item $\LEAK$,
      an execution latency measured in clock cycles
          (represented as a                       decimal integer string),
      and
\item $\PUB{r}$,
      a  result code
          (represented as a                       decimal integer string),
}%
Note that:

\begin{itemize}
\item The comparison between $\PRI{\CHAL{P}}$ and $\PUB{G}$ is implemented 
      by 
      $\PARTY{T}$ 
      as follows

      \begin{lstlisting}[language={C},gobble={6},frame={single},basicstyle={\ttfamily\small}]
      int match( const char* P, const char* G ) {
        int l_P = strlen( P );
        int l_G = strlen( G );
      
        if( l_P != l_G ) {
            return 0;
        }
      
        for( int i = 0; i < l_P; i++, Lambda = Lambda + 1 ) {
          if( P[ i ] != G[ i ] ) {
            return 0;
          }
        }
      
            return 1;
      }
      \end{lstlisting}

      \noindent
      which, in turn, implies that the result code is
      \[
      \VERB[C]{r} = \left\{\begin{array}{c l@{\;}c@{\;}c@{\;}c}
                           0 & \mbox{if} & \VERB[C]{P} &\neq& \VERB[C]{G} \\
                           1 & \mbox{if} & \VERB[C]{P} &=   & \VERB[C]{G} \\
                           \end{array}
                    \right.
      \]
\item Given the scenario, notice that even {\em without} the result code,
      $\PARTY{E}$ 
      would {\em still} learn whether or not a given attempt was correct: 
      if the attempt is incorrect then access to $\PARTY{T}$ is    allowed,
      whereas
      if the attempt is   correct then access                is disallowed.
\end{itemize}

\levelstay{\lstinline[language={bash}]|\$\{ARCHIVE\}/CONF(ARCHIVE_PATH,CID)/\$\{USER\}.R|}

\DESCMATERIALR{%
\item $\PUB{\REPL{G}}$,
      a  password attempt
      \IfStrEqCase{CONF(ALPHABET,CID)}{%
        {0}{%
          (represented as a  length-prefixed, hexadecimal octet   string),
        }%
        {1}{%
          (represented as an                              ASCII   string),
        }%
      }%
      and
\item $\PUB{\REPL{P}}$,
      a  password
      \IfStrEqCase{CONF(ALPHABET,CID)}{%
        {0}{%
          (represented as a  length-prefixed, hexadecimal octet   string),
        }%
        {1}{%
          (represented as an                              ASCII   string),
        }%
      }%
}%
In contrast to 
$\PARTY{T}$, 
this means 
$\PARTY{R}$ 
will use the
{\em chosen} password         $\PUB{\REPL{P}}$
for comparison with the
     chosen  password attempt $\PUB{\REPL{G}}$.

% -----------------------------------------------------------------------------

\levelup{Tasks}

      \DESCTASKIMPL
      {$\PRI{\CHAL{P}}$}
      {\mbox{\lstinline[language={bash}]|./attack \$\{USER\}.T|}}

% =============================================================================
